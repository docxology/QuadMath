% Options for packages loaded elsewhere
\PassOptionsToPackage{unicode}{hyperref}
\PassOptionsToPackage{hyphens}{url}
\PassOptionsToPackage{dvipsnames,svgnames*,x11names*}{xcolor}
%
\documentclass[
  10pt,
]{article}
\usepackage{lmodern}
\usepackage{setspace}
\usepackage{amssymb,amsmath}
\usepackage{ifxetex,ifluatex}
\ifnum 0\ifxetex 1\fi\ifluatex 1\fi=0 % if pdftex
  \usepackage[T1]{fontenc}
  \usepackage[utf8]{inputenc}
  \usepackage{textcomp} % provide euro and other symbols
\else % if luatex or xetex
  \usepackage{unicode-math}
  \defaultfontfeatures{Scale=MatchLowercase}
  \defaultfontfeatures[\rmfamily]{Ligatures=TeX,Scale=1}
  \setmainfont[]{DejaVu Serif}
  \setmonofont[]{DejaVu Sans Mono}
\fi
% Use upquote if available, for straight quotes in verbatim environments
\IfFileExists{upquote.sty}{\usepackage{upquote}}{}
\IfFileExists{microtype.sty}{% use microtype if available
  \usepackage[]{microtype}
  \UseMicrotypeSet[protrusion]{basicmath} % disable protrusion for tt fonts
}{}
\makeatletter
\@ifundefined{KOMAClassName}{% if non-KOMA class
  \IfFileExists{parskip.sty}{%
    \usepackage{parskip}
  }{% else
    \setlength{\parindent}{0pt}
    \setlength{\parskip}{6pt plus 2pt minus 1pt}}
}{% if KOMA class
  \KOMAoptions{parskip=half}}
\makeatother
\usepackage{xcolor}
\IfFileExists{xurl.sty}{\usepackage{xurl}}{} % add URL line breaks if available
\IfFileExists{bookmark.sty}{\usepackage{bookmark}}{\usepackage{hyperref}}
\hypersetup{
  colorlinks=true,
  linkcolor=red,
  filecolor=red,
  citecolor=red,
  urlcolor=red,
  pdfcreator={LaTeX via pandoc}}
\urlstyle{same} % disable monospaced font for URLs
\usepackage[margin=1cm,top=1cm,bottom=1cm,left=1cm,right=1cm,includeheadfoot]{geometry}
\usepackage{listings}
\newcommand{\passthrough}[1]{#1}
\lstset{defaultdialect=[5.3]Lua}
\lstset{defaultdialect=[x86masm]Assembler}
\usepackage{longtable,booktabs}
% Correct order of tables after \paragraph or \subparagraph
\usepackage{etoolbox}
\makeatletter
\patchcmd\longtable{\par}{\if@noskipsec\mbox{}\fi\par}{}{}
\makeatother
% Allow footnotes in longtable head/foot
\IfFileExists{footnotehyper.sty}{\usepackage{footnotehyper}}{\usepackage{footnote}}
\makesavenoteenv{longtable}
\setlength{\emergencystretch}{3em} % prevent overfull lines
\providecommand{\tightlist}{%
  \setlength{\itemsep}{0pt}\setlength{\parskip}{0pt}}
\setcounter{secnumdepth}{3}
% Enable graphics inclusion and ensure figure numbering works
\usepackage{graphicx}
\renewcommand{\figurename}{Figure}

% Configure fonts for Unicode support with fallbacks
\usepackage{newunicodechar}
\newunicodechar{⁴}{\textsuperscript{4}}
\newunicodechar{₄}{\textsubscript{4}}

% Enhanced code block styling for better contrast and readability
\usepackage{fancyvrb}
\usepackage{xcolor}
\usepackage{listings}

% Define custom colors for code blocks
\definecolor{codebg}{RGB}{245, 245, 245}      % Light gray background
\definecolor{codeborder}{RGB}{200, 200, 200}  % Medium gray border
\definecolor{codefg}{RGB}{50, 50, 50}         % Dark gray text

% Configure Verbatim environment for inline code
\DefineVerbatimEnvironment{Verbatim}{Verbatim}{%
    fontsize=\small,
    frame=single,
    framerule=0.5pt,
    framesep=3pt,
    rulecolor=\color{codeborder},
    bgcolor=\color{codebg},
    fgcolor=\color{codefg}
}

% Configure code block styling
\DefineVerbatimEnvironment{Highlighting}{Verbatim}{%
    fontsize=\footnotesize,
    frame=single,
    framerule=0.5pt,
    framesep=5pt,
    rulecolor=\color{codeborder},
    bgcolor=\color{codebg},
    fgcolor=\color{codefg}
}

% Style inline code with \texttt
\renewcommand{\texttt}[1]{%
    \colorbox{codebg}{\color{codefg}\ttfamily #1}%
}

% Configure listings package for code blocks
\lstset{
    backgroundcolor=\color{codebg},
    basicstyle=\footnotesize\ttfamily\color{codefg},
    breakatwhitespace=false,
    breaklines=true,
    captionpos=b,
    commentstyle=\color{codefg},
    deletekeywords={...},
    escapeinside={\%*}{*)},
    extendedchars=true,
    frame=single,
    framerule=0.5pt,
    framesep=5pt,
    keepspaces=true,
    keywordstyle=\color{codefg},
    language=Python,
    morekeywords={*,...},
    numbers=left,
    numbersep=5pt,
    numberstyle=\tiny\color{codefg},
    rulecolor=\color{codeborder},
    showspaces=false,
    showstringspaces=false,
    showtabs=false,
    stepnumber=1,
    stringstyle=\color{codefg},
    tabsize=2,
    title=\lstname
}

% Override any Pandoc default lstset configurations
\AtBeginDocument{
    \lstset{
        backgroundcolor=\color{codebg},
        basicstyle=\footnotesize\ttfamily\color{codefg},
        frame=single,
        framerule=0.5pt,
        framesep=5pt,
        rulecolor=\color{codeborder},
        numbers=left,
        numbersep=5pt,
        numberstyle=\tiny\color{codefg}
    }
}

% Configure hyperref colors consistently
\AtBeginDocument{
% Override pandoc's hidelinks setting with consistent options
\hypersetup{
    colorlinks=true,
    allcolors=red,
    linkcolor=red,
    urlcolor=red,
    citecolor=red,
    filecolor=red,
    menucolor=red,
    linktoc=all
}
}

% Simple page break support for document structure

\title{Appendix: Symbols and Glossary}
\author{Daniel Ari Friedman\\ ORCID: 0000-0001-6232-9096\\ Email: daniel@activeinference.institute}
\date{August 16, 2025}

\begin{document}
\maketitle

{
\hypersetup{linkcolor=black}
\setcounter{tocdepth}{3}
\tableofcontents
}
\setstretch{1.0}
\hypertarget{appendix-symbols-and-glossary}{%
\section{Appendix: Symbols and
Glossary}\label{appendix-symbols-and-glossary}}

This appendix consolidates the symbols, variables, and constants used
throughout the manuscript.

\hypertarget{sets-and-spaces}{%
\subsection{Sets and Spaces}\label{sets-and-spaces}}

\begin{longtable}[]{@{}ll@{}}
\toprule
Symbol & Name\tabularnewline
\midrule
\endhead
\(\mathbb{R}^n\) & Euclidean space\tabularnewline
IVM & Isotropic Vector Matrix\tabularnewline
Coxeter.4D & Euclidean 4D (E⁴)\tabularnewline
Einstein.4D & Minkowski spacetime (3+1)\tabularnewline
Fuller.4D & Synergetics/Quadray tetrahedral space\tabularnewline
\bottomrule
\end{longtable}

Descriptions:

\begin{itemize}
\tightlist
\item
  \(\mathbb{R}^n\): \(n\)-dimensional real vector space.
\item
  IVM: Quadray integer lattice (CCP sphere centers).
\item
  Coxeter.4D: Four-dimensional Euclidean geometry (not spacetime); see
  Coxeter, Regular Polytopes (Dover ed., p.~119); related
  lattice/packing background in Conway \& Sloane.
\item
  Einstein.4D: Relativistic spacetime with Minkowski metric.
\item
  Fuller.4D: Quadrays with projective normalization and IVM unit
  conventions.
\end{itemize}

\hypertarget{quadray-coordinates-and-geometry}{%
\subsection{Quadray Coordinates and
Geometry}\label{quadray-coordinates-and-geometry}}

\begin{longtable}[]{@{}lll@{}}
\toprule
\begin{minipage}[b]{0.30\columnwidth}\raggedright
Symbol\strut
\end{minipage} & \begin{minipage}[b]{0.30\columnwidth}\raggedright
Name\strut
\end{minipage} & \begin{minipage}[b]{0.30\columnwidth}\raggedright
Description\strut
\end{minipage}\tabularnewline
\midrule
\endhead
\begin{minipage}[t]{0.30\columnwidth}\raggedright
\(q=(a,b,c,d)\)\strut
\end{minipage} & \begin{minipage}[t]{0.30\columnwidth}\raggedright
Quadray point\strut
\end{minipage} & \begin{minipage}[t]{0.30\columnwidth}\raggedright
Non-negative coordinates with at least one zero after
normalization\strut
\end{minipage}\tabularnewline
\begin{minipage}[t]{0.30\columnwidth}\raggedright
\(A,B,C,D\)\strut
\end{minipage} & \begin{minipage}[t]{0.30\columnwidth}\raggedright
Quadray axes\strut
\end{minipage} & \begin{minipage}[t]{0.30\columnwidth}\raggedright
Canonical tetrahedral axes mapped by the embedding\strut
\end{minipage}\tabularnewline
\begin{minipage}[t]{0.30\columnwidth}\raggedright
\(k\)\strut
\end{minipage} & \begin{minipage}[t]{0.30\columnwidth}\raggedright
Normalization offset\strut
\end{minipage} & \begin{minipage}[t]{0.30\columnwidth}\raggedright
\(k=\min(a,b,c,d)\) used to set \(q' = q - (k,k,k,k)\)\strut
\end{minipage}\tabularnewline
\begin{minipage}[t]{0.30\columnwidth}\raggedright
\(q'\)\strut
\end{minipage} & \begin{minipage}[t]{0.30\columnwidth}\raggedright
Normalized Quadray\strut
\end{minipage} & \begin{minipage}[t]{0.30\columnwidth}\raggedright
Canonical representative with at least one zero and non-negative
entries\strut
\end{minipage}\tabularnewline
\begin{minipage}[t]{0.30\columnwidth}\raggedright
\(P_0,\ldots,P_3\)\strut
\end{minipage} & \begin{minipage}[t]{0.30\columnwidth}\raggedright
Tetrahedron vertices\strut
\end{minipage} & \begin{minipage}[t]{0.30\columnwidth}\raggedright
Vertices used in volume formulas\strut
\end{minipage}\tabularnewline
\begin{minipage}[t]{0.30\columnwidth}\raggedright
\(d_{ij}\)\strut
\end{minipage} & \begin{minipage}[t]{0.30\columnwidth}\raggedright
Pairwise distances\strut
\end{minipage} & \begin{minipage}[t]{0.30\columnwidth}\raggedright
Distance between vertices \(P_i\) and \(P_j\) (squared in CM
matrix)\strut
\end{minipage}\tabularnewline
\begin{minipage}[t]{0.30\columnwidth}\raggedright
\(\det(\cdot)\)\strut
\end{minipage} & \begin{minipage}[t]{0.30\columnwidth}\raggedright
Determinant\strut
\end{minipage} & \begin{minipage}[t]{0.30\columnwidth}\raggedright
Determinant of a matrix\strut
\end{minipage}\tabularnewline
\begin{minipage}[t]{0.30\columnwidth}\raggedright
\(\lvert\cdot\rvert\)\strut
\end{minipage} & \begin{minipage}[t]{0.30\columnwidth}\raggedright
Magnitude\strut
\end{minipage} & \begin{minipage}[t]{0.30\columnwidth}\raggedright
Absolute value (determinant magnitude)\strut
\end{minipage}\tabularnewline
\begin{minipage}[t]{0.30\columnwidth}\raggedright
\(V_{ivm}\)\strut
\end{minipage} & \begin{minipage}[t]{0.30\columnwidth}\raggedright
Tetravolume (IVM)\strut
\end{minipage} & \begin{minipage}[t]{0.30\columnwidth}\raggedright
Tetrahedron volume in synergetics/IVM units; unit regular tetra has
\(V_{ivm}=1\)\strut
\end{minipage}\tabularnewline
\begin{minipage}[t]{0.30\columnwidth}\raggedright
\(V_{xyz}\)\strut
\end{minipage} & \begin{minipage}[t]{0.30\columnwidth}\raggedright
Tetravolume (XYZ)\strut
\end{minipage} & \begin{minipage}[t]{0.30\columnwidth}\raggedright
Euclidean tetrahedron volume\strut
\end{minipage}\tabularnewline
\begin{minipage}[t]{0.30\columnwidth}\raggedright
\(S3\)\strut
\end{minipage} & \begin{minipage}[t]{0.30\columnwidth}\raggedright
Scale factor\strut
\end{minipage} & \begin{minipage}[t]{0.30\columnwidth}\raggedright
\(S3=\sqrt{9/8}\) with \(V_{ivm} = S3\,V_{xyz}\) (synergetics unit
convention)\strut
\end{minipage}\tabularnewline
\begin{minipage}[t]{0.30\columnwidth}\raggedright
Coxeter.4D\strut
\end{minipage} & \begin{minipage}[t]{0.30\columnwidth}\raggedright
Namespace\strut
\end{minipage} & \begin{minipage}[t]{0.30\columnwidth}\raggedright
Euclidean E⁴; regular polytopes\strut
\end{minipage}\tabularnewline
\begin{minipage}[t]{0.30\columnwidth}\raggedright
Einstein.4D\strut
\end{minipage} & \begin{minipage}[t]{0.30\columnwidth}\raggedright
Namespace\strut
\end{minipage} & \begin{minipage}[t]{0.30\columnwidth}\raggedright
Minkowski spacetime (metric analogy only here)\strut
\end{minipage}\tabularnewline
\begin{minipage}[t]{0.30\columnwidth}\raggedright
Fuller.4D\strut
\end{minipage} & \begin{minipage}[t]{0.30\columnwidth}\raggedright
Namespace\strut
\end{minipage} & \begin{minipage}[t]{0.30\columnwidth}\raggedright
Quadrays/IVM; integer tetravolume\strut
\end{minipage}\tabularnewline
\begin{minipage}[t]{0.30\columnwidth}\raggedright
Eq. (lattice\_det)\strut
\end{minipage} & \begin{minipage}[t]{0.30\columnwidth}\raggedright
Lattice determinant\strut
\end{minipage} & \begin{minipage}[t]{0.30\columnwidth}\raggedright
Integer-lattice volume via 3x3 determinant\strut
\end{minipage}\tabularnewline
\begin{minipage}[t]{0.30\columnwidth}\raggedright
Eq. (ace5x5)\strut
\end{minipage} & \begin{minipage}[t]{0.30\columnwidth}\raggedright
Tom Ace 5x5\strut
\end{minipage} & \begin{minipage}[t]{0.30\columnwidth}\raggedright
Direct IVM tetravolume from Quadrays\strut
\end{minipage}\tabularnewline
\begin{minipage}[t]{0.30\columnwidth}\raggedright
Eq. (cayley\_menger)\strut
\end{minipage} & \begin{minipage}[t]{0.30\columnwidth}\raggedright
Cayley--Menger\strut
\end{minipage} & \begin{minipage}[t]{0.30\columnwidth}\raggedright
Length-based formula: 288 V\^{}2 = det(·)\strut
\end{minipage}\tabularnewline
\bottomrule
\end{longtable}

\hypertarget{optimization-and-algorithms}{%
\subsection{Optimization and
Algorithms}\label{optimization-and-algorithms}}

\begin{longtable}[]{@{}ll@{}}
\toprule
Symbol & Name\tabularnewline
\midrule
\endhead
\(\alpha\) & Reflection coefficient\tabularnewline
\(\gamma\) & Expansion coefficient\tabularnewline
\(\rho\) & Contraction coefficient\tabularnewline
\(\sigma\) & Shrink coefficient\tabularnewline
\(V_{ivm}\) & Integer volume monitor\tabularnewline
\bottomrule
\end{longtable}

Descriptions:

\begin{itemize}
\tightlist
\item
  \(\alpha,\gamma,\rho,\sigma\): Nelder--Mead parameters (typical values
  1, 2, 0.5, 0.5).
\item
  \(V_{ivm}\): Tracks simplex volume across iterations.
\end{itemize}

\hypertarget{information-theory-and-geometry}{%
\subsection{Information Theory and
Geometry}\label{information-theory-and-geometry}}

\begin{longtable}[]{@{}lll@{}}
\toprule
\begin{minipage}[b]{0.30\columnwidth}\raggedright
Symbol\strut
\end{minipage} & \begin{minipage}[b]{0.30\columnwidth}\raggedright
Name\strut
\end{minipage} & \begin{minipage}[b]{0.30\columnwidth}\raggedright
Description\strut
\end{minipage}\tabularnewline
\midrule
\endhead
\begin{minipage}[t]{0.30\columnwidth}\raggedright
\(\log\)\strut
\end{minipage} & \begin{minipage}[t]{0.30\columnwidth}\raggedright
Natural logarithm\strut
\end{minipage} & \begin{minipage}[t]{0.30\columnwidth}\raggedright
Logarithm base \(e\)\strut
\end{minipage}\tabularnewline
\begin{minipage}[t]{0.30\columnwidth}\raggedright
\(\mathbb{E}[\cdot]\)\strut
\end{minipage} & \begin{minipage}[t]{0.30\columnwidth}\raggedright
Expectation\strut
\end{minipage} & \begin{minipage}[t]{0.30\columnwidth}\raggedright
Mean with respect to a distribution\strut
\end{minipage}\tabularnewline
\begin{minipage}[t]{0.30\columnwidth}\raggedright
\(F_{ij}\)\strut
\end{minipage} & \begin{minipage}[t]{0.30\columnwidth}\raggedright
Fisher Information Matrix\strut
\end{minipage} & \begin{minipage}[t]{0.30\columnwidth}\raggedright
\(\mathbb{E}[\partial_{\theta_i}\log p \cdot \partial_{\theta_j}\log p]\);
Eq. \eqref{eq:fim} in the equations appendix\strut
\end{minipage}\tabularnewline
\begin{minipage}[t]{0.30\columnwidth}\raggedright
\(\mathcal{F}\)\strut
\end{minipage} & \begin{minipage}[t]{0.30\columnwidth}\raggedright
Variational free energy\strut
\end{minipage} & \begin{minipage}[t]{0.30\columnwidth}\raggedright
\(-\log P(o\mid s) + \mathrm{KL}\big[Q(s)\,\|\,P(s)\big]\); Eq.
\eqref{eq:free_energy} in the equations appendix\strut
\end{minipage}\tabularnewline
\begin{minipage}[t]{0.30\columnwidth}\raggedright
\(\mathrm{KL}[Q\,\|\,P]\)\strut
\end{minipage} & \begin{minipage}[t]{0.30\columnwidth}\raggedright
Kullback--Leibler divergence\strut
\end{minipage} & \begin{minipage}[t]{0.30\columnwidth}\raggedright
\(\sum Q\log(Q/P)\); information distance\strut
\end{minipage}\tabularnewline
\begin{minipage}[t]{0.30\columnwidth}\raggedright
\(\nabla_{\theta} L\)\strut
\end{minipage} & \begin{minipage}[t]{0.30\columnwidth}\raggedright
Natural gradient\strut
\end{minipage} & \begin{minipage}[t]{0.30\columnwidth}\raggedright
\(F(\theta)^{-1} \nabla_{\theta} L(\theta)\); Eq.
\eqref{eq:natural_gradient} in the equations appendix\strut
\end{minipage}\tabularnewline
\begin{minipage}[t]{0.30\columnwidth}\raggedright
\(\eta\)\strut
\end{minipage} & \begin{minipage}[t]{0.30\columnwidth}\raggedright
Step size\strut
\end{minipage} & \begin{minipage}[t]{0.30\columnwidth}\raggedright
Learning-rate scalar used in updates\strut
\end{minipage}\tabularnewline
\begin{minipage}[t]{0.30\columnwidth}\raggedright
\(\theta\)\strut
\end{minipage} & \begin{minipage}[t]{0.30\columnwidth}\raggedright
Parameters\strut
\end{minipage} & \begin{minipage}[t]{0.30\columnwidth}\raggedright
Model parameter vector; indices \(\theta_i\)\strut
\end{minipage}\tabularnewline
\begin{minipage}[t]{0.30\columnwidth}\raggedright
\(ds^2\)\strut
\end{minipage} & \begin{minipage}[t]{0.30\columnwidth}\raggedright
Minkowski line element\strut
\end{minipage} & \begin{minipage}[t]{0.30\columnwidth}\raggedright
\(-c^2\,dt^2 + dx^2 + dy^2 + dz^2\); Eq.
\eqref{eq:minkowski_line_element} in the equations appendix\strut
\end{minipage}\tabularnewline
\begin{minipage}[t]{0.30\columnwidth}\raggedright
\(c\)\strut
\end{minipage} & \begin{minipage}[t]{0.30\columnwidth}\raggedright
Speed of light\strut
\end{minipage} & \begin{minipage}[t]{0.30\columnwidth}\raggedright
Physical constant appearing in Minkowski metric\strut
\end{minipage}\tabularnewline
\bottomrule
\end{longtable}

\hypertarget{embeddings-and-distances}{%
\subsection{Embeddings and Distances}\label{embeddings-and-distances}}

\begin{longtable}[]{@{}lll@{}}
\toprule
\begin{minipage}[b]{0.30\columnwidth}\raggedright
Symbol\strut
\end{minipage} & \begin{minipage}[b]{0.30\columnwidth}\raggedright
Name\strut
\end{minipage} & \begin{minipage}[b]{0.30\columnwidth}\raggedright
Description\strut
\end{minipage}\tabularnewline
\midrule
\endhead
\begin{minipage}[t]{0.30\columnwidth}\raggedright
\(M\)\strut
\end{minipage} & \begin{minipage}[t]{0.30\columnwidth}\raggedright
Embedding matrix\strut
\end{minipage} & \begin{minipage}[t]{0.30\columnwidth}\raggedright
Linear map from Quadray to \(\mathbb{R}^3\) (Urner-style unless
noted)\strut
\end{minipage}\tabularnewline
\begin{minipage}[t]{0.30\columnwidth}\raggedright
\(\lVert\cdot\rVert_2\)\strut
\end{minipage} & \begin{minipage}[t]{0.30\columnwidth}\raggedright
Euclidean norm\strut
\end{minipage} & \begin{minipage}[t]{0.30\columnwidth}\raggedright
\(\sqrt{x_1^2+\cdots+x_n^2}\)\strut
\end{minipage}\tabularnewline
\begin{minipage}[t]{0.30\columnwidth}\raggedright
\(R, D\)\strut
\end{minipage} & \begin{minipage}[t]{0.30\columnwidth}\raggedright
Edge scales\strut
\end{minipage} & \begin{minipage}[t]{0.30\columnwidth}\raggedright
Cube edge \(R\) and Quadray edge \(D\) with \(D=2R\) (common
convention)\strut
\end{minipage}\tabularnewline
\bottomrule
\end{longtable}

\hypertarget{greek-letters-usage}{%
\subsection{Greek Letters (usage)}\label{greek-letters-usage}}

\begin{longtable}[]{@{}lll@{}}
\toprule
\begin{minipage}[b]{0.30\columnwidth}\raggedright
Symbol\strut
\end{minipage} & \begin{minipage}[b]{0.30\columnwidth}\raggedright
Name\strut
\end{minipage} & \begin{minipage}[b]{0.30\columnwidth}\raggedright
Description\strut
\end{minipage}\tabularnewline
\midrule
\endhead
\begin{minipage}[t]{0.30\columnwidth}\raggedright
\(\alpha,\gamma,\rho,\sigma\)\strut
\end{minipage} & \begin{minipage}[t]{0.30\columnwidth}\raggedright
NM coefficients\strut
\end{minipage} & \begin{minipage}[t]{0.30\columnwidth}\raggedright
Nelder--Mead parameters (reflection, expansion, contraction,
shrink)\strut
\end{minipage}\tabularnewline
\begin{minipage}[t]{0.30\columnwidth}\raggedright
\(\theta\)\strut
\end{minipage} & \begin{minipage}[t]{0.30\columnwidth}\raggedright
Theta\strut
\end{minipage} & \begin{minipage}[t]{0.30\columnwidth}\raggedright
Parameter vector in models and metrics\strut
\end{minipage}\tabularnewline
\begin{minipage}[t]{0.30\columnwidth}\raggedright
\(\mu\)\strut
\end{minipage} & \begin{minipage}[t]{0.30\columnwidth}\raggedright
Mu\strut
\end{minipage} & \begin{minipage}[t]{0.30\columnwidth}\raggedright
Internal states (Active Inference)\strut
\end{minipage}\tabularnewline
\begin{minipage}[t]{0.30\columnwidth}\raggedright
\(\psi\)\strut
\end{minipage} & \begin{minipage}[t]{0.30\columnwidth}\raggedright
Psi\strut
\end{minipage} & \begin{minipage}[t]{0.30\columnwidth}\raggedright
External states (Active Inference)\strut
\end{minipage}\tabularnewline
\begin{minipage}[t]{0.30\columnwidth}\raggedright
\(\eta\)\strut
\end{minipage} & \begin{minipage}[t]{0.30\columnwidth}\raggedright
Eta\strut
\end{minipage} & \begin{minipage}[t]{0.30\columnwidth}\raggedright
Step size / learning rate\strut
\end{minipage}\tabularnewline
\bottomrule
\end{longtable}

\hypertarget{notes-usage-and-cross-references}{%
\subsection{Notes (usage and
cross-references)}\label{notes-usage-and-cross-references}}

\begin{itemize}
\tightlist
\item
  \textbf{Figures referenced}: In-text references use LaTeX's automatic
  figure numbering for consistent cross-referencing.
\item
  \textbf{Equation references}: Use labels defined in the text (e.g.,
  Eq. \eqref{eq:lattice_det} in the equations appendix).
\item
  \textbf{Namespaces}: We use Coxeter.4D, Einstein.4D, Fuller.4D
  consistently to designate Euclidean E⁴, Minkowski spacetime, and
  Quadray/IVM synergetics, respectively. This avoids conflation of
  Euclidean 4D objects (e.g., tesseracts) with spacetime constructs and
  synergetic tetravolume conventions.
\item
  \textbf{External validation}: Cross-reference implementations from the
  \href{https://github.com/4dsolutions}{4dsolutions ecosystem} for
  algorithmic verification and performance comparison baselines. See the
  \href{07_resources.md}{Resources} section for comprehensive details.
\end{itemize}

\hypertarget{polyhedra-and-synergetic-shapes}{%
\subsection{Polyhedra and Synergetic
Shapes}\label{polyhedra-and-synergetic-shapes}}

\begin{longtable}[]{@{}lll@{}}
\toprule
\begin{minipage}[b]{0.30\columnwidth}\raggedright
Symbol\strut
\end{minipage} & \begin{minipage}[b]{0.30\columnwidth}\raggedright
Name\strut
\end{minipage} & \begin{minipage}[b]{0.30\columnwidth}\raggedright
Description\strut
\end{minipage}\tabularnewline
\midrule
\endhead
\begin{minipage}[t]{0.30\columnwidth}\raggedright
Tetrahedron\strut
\end{minipage} & \begin{minipage}[t]{0.30\columnwidth}\raggedright
Regular tetrahedron\strut
\end{minipage} & \begin{minipage}[t]{0.30\columnwidth}\raggedright
Fundamental unit with V=1 in IVM units\strut
\end{minipage}\tabularnewline
\begin{minipage}[t]{0.30\columnwidth}\raggedright
Cube\strut
\end{minipage} & \begin{minipage}[t]{0.30\columnwidth}\raggedright
Regular hexahedron\strut
\end{minipage} & \begin{minipage}[t]{0.30\columnwidth}\raggedright
V=3 in IVM units; orthogonal space-filling\strut
\end{minipage}\tabularnewline
\begin{minipage}[t]{0.30\columnwidth}\raggedright
Octahedron\strut
\end{minipage} & \begin{minipage}[t]{0.30\columnwidth}\raggedright
Regular octahedron\strut
\end{minipage} & \begin{minipage}[t]{0.30\columnwidth}\raggedright
V=4 in IVM units; edge-midpoint construction\strut
\end{minipage}\tabularnewline
\begin{minipage}[t]{0.30\columnwidth}\raggedright
Rhombic Dodecahedron\strut
\end{minipage} & \begin{minipage}[t]{0.30\columnwidth}\raggedright
12-faced solid\strut
\end{minipage} & \begin{minipage}[t]{0.30\columnwidth}\raggedright
V=6 in IVM units; Voronoi cell of FCC packing\strut
\end{minipage}\tabularnewline
\begin{minipage}[t]{0.30\columnwidth}\raggedright
Cuboctahedron\strut
\end{minipage} & \begin{minipage}[t]{0.30\columnwidth}\raggedright
Vector equilibrium\strut
\end{minipage} & \begin{minipage}[t]{0.30\columnwidth}\raggedright
V=20 in IVM units; shell of 12 IVM neighbors\strut
\end{minipage}\tabularnewline
\begin{minipage}[t]{0.30\columnwidth}\raggedright
Truncated Octahedron\strut
\end{minipage} & \begin{minipage}[t]{0.30\columnwidth}\raggedright
Archimedean solid\strut
\end{minipage} & \begin{minipage}[t]{0.30\columnwidth}\raggedright
V=20 in IVM units; space-filling tiling\strut
\end{minipage}\tabularnewline
\bottomrule
\end{longtable}

\hypertarget{acronyms-and-abbreviations}{%
\subsection{Acronyms and
abbreviations}\label{acronyms-and-abbreviations}}

\begin{longtable}[]{@{}ll@{}}
\toprule
\begin{minipage}[b]{0.47\columnwidth}\raggedright
Acronym\strut
\end{minipage} & \begin{minipage}[b]{0.47\columnwidth}\raggedright
Meaning\strut
\end{minipage}\tabularnewline
\midrule
\endhead
\begin{minipage}[t]{0.47\columnwidth}\raggedright
CM\strut
\end{minipage} & \begin{minipage}[t]{0.47\columnwidth}\raggedright
Cayley--Menger (determinant-based tetrahedron volume)\strut
\end{minipage}\tabularnewline
\begin{minipage}[t]{0.47\columnwidth}\raggedright
PdF\strut
\end{minipage} & \begin{minipage}[t]{0.47\columnwidth}\raggedright
Piero della Francesca (Heron-like tetrahedron volume)\strut
\end{minipage}\tabularnewline
\begin{minipage}[t]{0.47\columnwidth}\raggedright
GdJ\strut
\end{minipage} & \begin{minipage}[t]{0.47\columnwidth}\raggedright
Gerald de Jong (Quadray-native tetravolume expression)\strut
\end{minipage}\tabularnewline
\begin{minipage}[t]{0.47\columnwidth}\raggedright
K-FAC\strut
\end{minipage} & \begin{minipage}[t]{0.47\columnwidth}\raggedright
Kronecker-Factored Approximate Curvature (optimizer using structured
Fisher)\strut
\end{minipage}\tabularnewline
\begin{minipage}[t]{0.47\columnwidth}\raggedright
CCP\strut
\end{minipage} & \begin{minipage}[t]{0.47\columnwidth}\raggedright
Cubic Close Packing (same centers as FCC)\strut
\end{minipage}\tabularnewline
\begin{minipage}[t]{0.47\columnwidth}\raggedright
FCC\strut
\end{minipage} & \begin{minipage}[t]{0.47\columnwidth}\raggedright
Face-Centered Cubic (same centers as CCP)\strut
\end{minipage}\tabularnewline
\begin{minipage}[t]{0.47\columnwidth}\raggedright
E⁴\strut
\end{minipage} & \begin{minipage}[t]{0.47\columnwidth}\raggedright
Four-dimensional Euclidean space (Coxeter.4D)\strut
\end{minipage}\tabularnewline
\begin{minipage}[t]{0.47\columnwidth}\raggedright
NM\strut
\end{minipage} & \begin{minipage}[t]{0.47\columnwidth}\raggedright
Nelder--Mead (simplex optimization algorithm)\strut
\end{minipage}\tabularnewline
\begin{minipage}[t]{0.47\columnwidth}\raggedright
4dsolutions\strut
\end{minipage} & \begin{minipage}[t]{0.47\columnwidth}\raggedright
Kirby Urner's GitHub organization with extensive Quadray
implementations\strut
\end{minipage}\tabularnewline
\begin{minipage}[t]{0.47\columnwidth}\raggedright
BEAST\strut
\end{minipage} & \begin{minipage}[t]{0.47\columnwidth}\raggedright
Synergetic modules (B, E, A, S, T) in Fuller's hierarchical system\strut
\end{minipage}\tabularnewline
\begin{minipage}[t]{0.47\columnwidth}\raggedright
OCN\strut
\end{minipage} & \begin{minipage}[t]{0.47\columnwidth}\raggedright
Oregon Curriculum Network (educational framework integrating
Quadrays)\strut
\end{minipage}\tabularnewline
\begin{minipage}[t]{0.47\columnwidth}\raggedright
POV-Ray\strut
\end{minipage} & \begin{minipage}[t]{0.47\columnwidth}\raggedright
Persistence of Vision Raytracer (used in quadcraft.py
visualizations)\strut
\end{minipage}\tabularnewline
\bottomrule
\end{longtable}

\hypertarget{api-index-auto-generated-methods-linkage}{%
\subsection{API Index (auto-generated; Methods
linkage)}\label{api-index-auto-generated-methods-linkage}}

The table below enumerates public symbols from
\passthrough{\lstinline!src/!} modules.

\begin{longtable}[]{@{}lllll@{}}
\toprule
\begin{minipage}[b]{0.17\columnwidth}\raggedright
Module\strut
\end{minipage} & \begin{minipage}[b]{0.17\columnwidth}\raggedright
Symbol\strut
\end{minipage} & \begin{minipage}[b]{0.17\columnwidth}\raggedright
Kind\strut
\end{minipage} & \begin{minipage}[b]{0.17\columnwidth}\raggedright
Signature\strut
\end{minipage} & \begin{minipage}[b]{0.17\columnwidth}\raggedright
Summary\strut
\end{minipage}\tabularnewline
\midrule
\endhead
\begin{minipage}[t]{0.17\columnwidth}\raggedright
\passthrough{\lstinline!cayley\_menger!}\strut
\end{minipage} & \begin{minipage}[t]{0.17\columnwidth}\raggedright
\passthrough{\lstinline!ivm\_tetra\_volume\_cayley\_menger!}\strut
\end{minipage} & \begin{minipage}[t]{0.17\columnwidth}\raggedright
function\strut
\end{minipage} & \begin{minipage}[t]{0.17\columnwidth}\raggedright
\passthrough{\lstinline!(d2)!}\strut
\end{minipage} & \begin{minipage}[t]{0.17\columnwidth}\raggedright
Compute IVM tetravolume from squared distances via Cayley--Menger.\strut
\end{minipage}\tabularnewline
\begin{minipage}[t]{0.17\columnwidth}\raggedright
\passthrough{\lstinline!cayley\_menger!}\strut
\end{minipage} & \begin{minipage}[t]{0.17\columnwidth}\raggedright
\passthrough{\lstinline!tetra\_volume\_cayley\_menger!}\strut
\end{minipage} & \begin{minipage}[t]{0.17\columnwidth}\raggedright
function\strut
\end{minipage} & \begin{minipage}[t]{0.17\columnwidth}\raggedright
\passthrough{\lstinline!(d2)!}\strut
\end{minipage} & \begin{minipage}[t]{0.17\columnwidth}\raggedright
Compute Euclidean tetrahedron volume from squared distances
(Coxeter.4D).\strut
\end{minipage}\tabularnewline
\begin{minipage}[t]{0.17\columnwidth}\raggedright
\passthrough{\lstinline!conversions!}\strut
\end{minipage} & \begin{minipage}[t]{0.17\columnwidth}\raggedright
\passthrough{\lstinline!quadray\_to\_xyz!}\strut
\end{minipage} & \begin{minipage}[t]{0.17\columnwidth}\raggedright
function\strut
\end{minipage} & \begin{minipage}[t]{0.17\columnwidth}\raggedright
\passthrough{\lstinline!(q, M)!}\strut
\end{minipage} & \begin{minipage}[t]{0.17\columnwidth}\raggedright
Map a \passthrough{\lstinline!Quadray!} to Cartesian XYZ via a 3x4
embedding matrix (Fuller.4D -\textgreater{} Coxeter.4D slice).\strut
\end{minipage}\tabularnewline
\begin{minipage}[t]{0.17\columnwidth}\raggedright
\passthrough{\lstinline!conversions!}\strut
\end{minipage} & \begin{minipage}[t]{0.17\columnwidth}\raggedright
\passthrough{\lstinline!urner\_embedding!}\strut
\end{minipage} & \begin{minipage}[t]{0.17\columnwidth}\raggedright
function\strut
\end{minipage} & \begin{minipage}[t]{0.17\columnwidth}\raggedright
\passthrough{\lstinline!(scale)!}\strut
\end{minipage} & \begin{minipage}[t]{0.17\columnwidth}\raggedright
Return a 3x4 Urner-style symmetric embedding matrix (Fuller.4D
-\textgreater{} Coxeter.4D slice).\strut
\end{minipage}\tabularnewline
\begin{minipage}[t]{0.17\columnwidth}\raggedright
\passthrough{\lstinline!discrete\_variational!}\strut
\end{minipage} & \begin{minipage}[t]{0.17\columnwidth}\raggedright
\passthrough{\lstinline!DiscretePath!}\strut
\end{minipage} & \begin{minipage}[t]{0.17\columnwidth}\raggedright
class\strut
\end{minipage} & \begin{minipage}[t]{0.17\columnwidth}\raggedright
\passthrough{\lstinline!| Optimization trajectory on the integer quadray lattice. | | `discrete\_variational` | `OptionalMoves` | class |!}\strut
\end{minipage} & \begin{minipage}[t]{0.17\columnwidth}\raggedright
\strut
\end{minipage}\tabularnewline
\begin{minipage}[t]{0.17\columnwidth}\raggedright
\passthrough{\lstinline!discrete\_variational!}\strut
\end{minipage} & \begin{minipage}[t]{0.17\columnwidth}\raggedright
\passthrough{\lstinline!apply\_move!}\strut
\end{minipage} & \begin{minipage}[t]{0.17\columnwidth}\raggedright
function\strut
\end{minipage} & \begin{minipage}[t]{0.17\columnwidth}\raggedright
\passthrough{\lstinline!(q, delta)!}\strut
\end{minipage} & \begin{minipage}[t]{0.17\columnwidth}\raggedright
Apply a lattice move and normalize to the canonical
representative.\strut
\end{minipage}\tabularnewline
\begin{minipage}[t]{0.17\columnwidth}\raggedright
\passthrough{\lstinline!discrete\_variational!}\strut
\end{minipage} & \begin{minipage}[t]{0.17\columnwidth}\raggedright
\passthrough{\lstinline!discrete\_ivm\_descent!}\strut
\end{minipage} & \begin{minipage}[t]{0.17\columnwidth}\raggedright
function\strut
\end{minipage} & \begin{minipage}[t]{0.17\columnwidth}\raggedright
\passthrough{\lstinline!(objective, start, moves=, max\_iter=, on\_step=)!}\strut
\end{minipage} & \begin{minipage}[t]{0.17\columnwidth}\raggedright
Greedy discrete descent over the quadray integer lattice.\strut
\end{minipage}\tabularnewline
\begin{minipage}[t]{0.17\columnwidth}\raggedright
\passthrough{\lstinline!discrete\_variational!}\strut
\end{minipage} & \begin{minipage}[t]{0.17\columnwidth}\raggedright
\passthrough{\lstinline!neighbor\_moves\_ivm!}\strut
\end{minipage} & \begin{minipage}[t]{0.17\columnwidth}\raggedright
function\strut
\end{minipage} & \begin{minipage}[t]{0.17\columnwidth}\raggedright
\passthrough{\lstinline!()!}\strut
\end{minipage} & \begin{minipage}[t]{0.17\columnwidth}\raggedright
Return the 12 canonical IVM neighbor moves as Quadray deltas.\strut
\end{minipage}\tabularnewline
\begin{minipage}[t]{0.17\columnwidth}\raggedright
\passthrough{\lstinline!examples!}\strut
\end{minipage} & \begin{minipage}[t]{0.17\columnwidth}\raggedright
\passthrough{\lstinline!example\_cuboctahedron\_neighbors!}\strut
\end{minipage} & \begin{minipage}[t]{0.17\columnwidth}\raggedright
function\strut
\end{minipage} & \begin{minipage}[t]{0.17\columnwidth}\raggedright
\passthrough{\lstinline!()!}\strut
\end{minipage} & \begin{minipage}[t]{0.17\columnwidth}\raggedright
Return twelve-around-one IVM neighbors (vector equilibrium shell).\strut
\end{minipage}\tabularnewline
\begin{minipage}[t]{0.17\columnwidth}\raggedright
\passthrough{\lstinline!examples!}\strut
\end{minipage} & \begin{minipage}[t]{0.17\columnwidth}\raggedright
\passthrough{\lstinline!example\_cuboctahedron\_vertices\_xyz!}\strut
\end{minipage} & \begin{minipage}[t]{0.17\columnwidth}\raggedright
function\strut
\end{minipage} & \begin{minipage}[t]{0.17\columnwidth}\raggedright
\passthrough{\lstinline!()!}\strut
\end{minipage} & \begin{minipage}[t]{0.17\columnwidth}\raggedright
Return XYZ coordinates for the twelve-around-one neighbors.\strut
\end{minipage}\tabularnewline
\begin{minipage}[t]{0.17\columnwidth}\raggedright
\passthrough{\lstinline!examples!}\strut
\end{minipage} & \begin{minipage}[t]{0.17\columnwidth}\raggedright
\passthrough{\lstinline!example\_ivm\_neighbors!}\strut
\end{minipage} & \begin{minipage}[t]{0.17\columnwidth}\raggedright
function\strut
\end{minipage} & \begin{minipage}[t]{0.17\columnwidth}\raggedright
\passthrough{\lstinline!()!}\strut
\end{minipage} & \begin{minipage}[t]{0.17\columnwidth}\raggedright
Return the 12 nearest IVM neighbors as permutations of \{2,1,1,0\}
(Fuller.4D).\strut
\end{minipage}\tabularnewline
\begin{minipage}[t]{0.17\columnwidth}\raggedright
\passthrough{\lstinline!examples!}\strut
\end{minipage} & \begin{minipage}[t]{0.17\columnwidth}\raggedright
\passthrough{\lstinline!example\_optimize!}\strut
\end{minipage} & \begin{minipage}[t]{0.17\columnwidth}\raggedright
function\strut
\end{minipage} & \begin{minipage}[t]{0.17\columnwidth}\raggedright
\passthrough{\lstinline!()!}\strut
\end{minipage} & \begin{minipage}[t]{0.17\columnwidth}\raggedright
Run Nelder--Mead over integer quadrays for a simple convex objective
(Fuller.4D).\strut
\end{minipage}\tabularnewline
\begin{minipage}[t]{0.17\columnwidth}\raggedright
\passthrough{\lstinline!examples!}\strut
\end{minipage} & \begin{minipage}[t]{0.17\columnwidth}\raggedright
\passthrough{\lstinline!example\_partition\_tetra\_volume!}\strut
\end{minipage} & \begin{minipage}[t]{0.17\columnwidth}\raggedright
function\strut
\end{minipage} & \begin{minipage}[t]{0.17\columnwidth}\raggedright
\passthrough{\lstinline!(mu, s, a, psi)!}\strut
\end{minipage} & \begin{minipage}[t]{0.17\columnwidth}\raggedright
Construct a tetrahedron from the four-fold partition and return
tetravolume (Fuller.4D).\strut
\end{minipage}\tabularnewline
\begin{minipage}[t]{0.17\columnwidth}\raggedright
\passthrough{\lstinline!examples!}\strut
\end{minipage} & \begin{minipage}[t]{0.17\columnwidth}\raggedright
\passthrough{\lstinline!example\_volume!}\strut
\end{minipage} & \begin{minipage}[t]{0.17\columnwidth}\raggedright
function\strut
\end{minipage} & \begin{minipage}[t]{0.17\columnwidth}\raggedright
\passthrough{\lstinline!()!}\strut
\end{minipage} & \begin{minipage}[t]{0.17\columnwidth}\raggedright
Compute the unit IVM tetrahedron volume from simple quadray vertices
(Fuller.4D).\strut
\end{minipage}\tabularnewline
\begin{minipage}[t]{0.17\columnwidth}\raggedright
\passthrough{\lstinline!geometry!}\strut
\end{minipage} & \begin{minipage}[t]{0.17\columnwidth}\raggedright
\passthrough{\lstinline!minkowski\_interval!}\strut
\end{minipage} & \begin{minipage}[t]{0.17\columnwidth}\raggedright
function\strut
\end{minipage} & \begin{minipage}[t]{0.17\columnwidth}\raggedright
\passthrough{\lstinline!(dt, dx, dy, dz, c)!}\strut
\end{minipage} & \begin{minipage}[t]{0.17\columnwidth}\raggedright
Return the Minkowski interval squared ds\^{}2 (Einstein.4D).\strut
\end{minipage}\tabularnewline
\begin{minipage}[t]{0.17\columnwidth}\raggedright
\passthrough{\lstinline!glossary\_gen!}\strut
\end{minipage} & \begin{minipage}[t]{0.17\columnwidth}\raggedright
\passthrough{\lstinline!ApiEntry!}\strut
\end{minipage} & \begin{minipage}[t]{0.17\columnwidth}\raggedright
class\strut
\end{minipage} & \begin{minipage}[t]{0.17\columnwidth}\raggedright
\passthrough{\lstinline!|  | | `glossary\_gen` | `build\_api\_index` | function | `(src\_dir)` |  | | `glossary\_gen` | `generate\_markdown\_table` | function | `(entries)` |  | | `glossary\_gen` | `inject\_between\_markers` | function | `(markdown\_text, begin, end, payload)` |  | | `information` | `action\_update` | function | `(action, free\_energy\_fn, step\_size, epsilon)` | Continuous-time action update: da/dt = - dF/da. | | `information` | `active\_inference\_step` | function | `(mu, action, free\_energy\_fn, derivative\_operator, step\_size, epsilon)` | Joint perception-action update step in Active Inference. | | `information` | `expected\_free\_energy` | function | `(log\_p\_o\_given\_s, q, p, log\_p\_o)` | Expected free energy for Active Inference with prior preferences. | | `information` | `finite\_difference\_gradient` | function | `(function, x, epsilon)` | Compute numerical gradient of a scalar function via central differences. | | `information` | `fisher\_information\_matrix` | function | `(gradients, normalize)` | Estimate the Fisher information matrix via sample gradients. | | `information` | `fisher\_information\_quadray` | function | `(gradients, embedding\_matrix)` | Compute Fisher information matrix in both Cartesian and Quadray coordinates. | | `information` | `free\_energy` | function | `(log\_p\_o\_given\_s, q, p)` | Variational free energy for discrete latent states. | | `information` | `information\_geometric\_distance` | function | `(F, x1, x2)` | Compute information-geometric distance between two points. | | `information` | `natural\_gradient\_step` | function | `(gradient, fisher, step\_size, ridge)` | Compute a natural gradient step using a damped inverse Fisher. | | `information` | `perception\_update` | function | `(mu, derivative\_operator, free\_energy\_fn, step\_size, epsilon)` | Continuous-time perception update: dmu/dt = D mu - dF/dmu. | | `linalg\_utils` | `bareiss\_determinant\_int` | function | `(matrix)` | Compute an exact integer determinant using the Bareiss algorithm. | | `metrics` | `fim\_eigenspectrum` | function | `(F)` | Eigen-decomposition of a Fisher information matrix. | | `metrics` | `fisher\_condition\_number` | function | `(F)` | Compute the condition number of the Fisher information matrix. | | `metrics` | `fisher\_curvature\_analysis` | function | `(F)` | Comprehensive analysis of Fisher information matrix curvature. | | `metrics` | `fisher\_quadray\_comparison` | function | `(F\_cartesian, F\_quadray)` | Compare Fisher information matrices between coordinate systems. | | `metrics` | `information\_length` | function | `(path\_gradients)` | Path length in information space via gradient-weighted arc length. | | `metrics` | `shannon\_entropy` | function | `(p, eps)` | Shannon entropy H(p) for a discrete distribution. | | `nelder\_mead\_quadray` | `SimplexState` | class |!}\strut
\end{minipage} & \begin{minipage}[t]{0.17\columnwidth}\raggedright
\strut
\end{minipage}\tabularnewline
\begin{minipage}[t]{0.17\columnwidth}\raggedright
\passthrough{\lstinline!nelder\_mead\_quadray!}\strut
\end{minipage} & \begin{minipage}[t]{0.17\columnwidth}\raggedright
\passthrough{\lstinline!centroid\_excluding!}\strut
\end{minipage} & \begin{minipage}[t]{0.17\columnwidth}\raggedright
function\strut
\end{minipage} & \begin{minipage}[t]{0.17\columnwidth}\raggedright
\passthrough{\lstinline!(vertices, exclude\_idx)!}\strut
\end{minipage} & \begin{minipage}[t]{0.17\columnwidth}\raggedright
Integer centroid of three vertices, excluding the specified index.\strut
\end{minipage}\tabularnewline
\begin{minipage}[t]{0.17\columnwidth}\raggedright
\passthrough{\lstinline!nelder\_mead\_quadray!}\strut
\end{minipage} & \begin{minipage}[t]{0.17\columnwidth}\raggedright
\passthrough{\lstinline!compute\_volume!}\strut
\end{minipage} & \begin{minipage}[t]{0.17\columnwidth}\raggedright
function\strut
\end{minipage} & \begin{minipage}[t]{0.17\columnwidth}\raggedright
\passthrough{\lstinline!(vertices)!}\strut
\end{minipage} & \begin{minipage}[t]{0.17\columnwidth}\raggedright
Integer IVM tetra-volume from the first four vertices.\strut
\end{minipage}\tabularnewline
\begin{minipage}[t]{0.17\columnwidth}\raggedright
\passthrough{\lstinline!nelder\_mead\_quadray!}\strut
\end{minipage} & \begin{minipage}[t]{0.17\columnwidth}\raggedright
\passthrough{\lstinline!nelder\_mead\_quadray!}\strut
\end{minipage} & \begin{minipage}[t]{0.17\columnwidth}\raggedright
function\strut
\end{minipage} & \begin{minipage}[t]{0.17\columnwidth}\raggedright
\passthrough{\lstinline!(f, initial\_vertices, alpha, gamma, rho, sigma, max\_iter, tol, on\_step)!}\strut
\end{minipage} & \begin{minipage}[t]{0.17\columnwidth}\raggedright
Nelder--Mead on the integer quadray lattice.\strut
\end{minipage}\tabularnewline
\begin{minipage}[t]{0.17\columnwidth}\raggedright
\passthrough{\lstinline!nelder\_mead\_quadray!}\strut
\end{minipage} & \begin{minipage}[t]{0.17\columnwidth}\raggedright
\passthrough{\lstinline!order\_simplex!}\strut
\end{minipage} & \begin{minipage}[t]{0.17\columnwidth}\raggedright
function\strut
\end{minipage} & \begin{minipage}[t]{0.17\columnwidth}\raggedright
\passthrough{\lstinline!(vertices, f)!}\strut
\end{minipage} & \begin{minipage}[t]{0.17\columnwidth}\raggedright
Sort vertices by objective value ascending and return paired
lists.\strut
\end{minipage}\tabularnewline
\begin{minipage}[t]{0.17\columnwidth}\raggedright
\passthrough{\lstinline!nelder\_mead\_quadray!}\strut
\end{minipage} & \begin{minipage}[t]{0.17\columnwidth}\raggedright
\passthrough{\lstinline!project\_to\_lattice!}\strut
\end{minipage} & \begin{minipage}[t]{0.17\columnwidth}\raggedright
function\strut
\end{minipage} & \begin{minipage}[t]{0.17\columnwidth}\raggedright
\passthrough{\lstinline!(q)!}\strut
\end{minipage} & \begin{minipage}[t]{0.17\columnwidth}\raggedright
Project a quadray to the canonical lattice representative via
normalize.\strut
\end{minipage}\tabularnewline
\begin{minipage}[t]{0.17\columnwidth}\raggedright
\passthrough{\lstinline!paths!}\strut
\end{minipage} & \begin{minipage}[t]{0.17\columnwidth}\raggedright
\passthrough{\lstinline!get\_data\_dir!}\strut
\end{minipage} & \begin{minipage}[t]{0.17\columnwidth}\raggedright
function\strut
\end{minipage} & \begin{minipage}[t]{0.17\columnwidth}\raggedright
\passthrough{\lstinline!()!}\strut
\end{minipage} & \begin{minipage}[t]{0.17\columnwidth}\raggedright
Return \passthrough{\lstinline!quadmath/output/data!} path and ensure it
exists.\strut
\end{minipage}\tabularnewline
\begin{minipage}[t]{0.17\columnwidth}\raggedright
\passthrough{\lstinline!paths!}\strut
\end{minipage} & \begin{minipage}[t]{0.17\columnwidth}\raggedright
\passthrough{\lstinline!get\_figure\_dir!}\strut
\end{minipage} & \begin{minipage}[t]{0.17\columnwidth}\raggedright
function\strut
\end{minipage} & \begin{minipage}[t]{0.17\columnwidth}\raggedright
\passthrough{\lstinline!()!}\strut
\end{minipage} & \begin{minipage}[t]{0.17\columnwidth}\raggedright
Return \passthrough{\lstinline!quadmath/output/figures!} path and ensure
it exists.\strut
\end{minipage}\tabularnewline
\begin{minipage}[t]{0.17\columnwidth}\raggedright
\passthrough{\lstinline!paths!}\strut
\end{minipage} & \begin{minipage}[t]{0.17\columnwidth}\raggedright
\passthrough{\lstinline!get\_output\_dir!}\strut
\end{minipage} & \begin{minipage}[t]{0.17\columnwidth}\raggedright
function\strut
\end{minipage} & \begin{minipage}[t]{0.17\columnwidth}\raggedright
\passthrough{\lstinline!()!}\strut
\end{minipage} & \begin{minipage}[t]{0.17\columnwidth}\raggedright
Return \passthrough{\lstinline!quadmath/output!} path at the repo root
and ensure it exists.\strut
\end{minipage}\tabularnewline
\begin{minipage}[t]{0.17\columnwidth}\raggedright
\passthrough{\lstinline!paths!}\strut
\end{minipage} & \begin{minipage}[t]{0.17\columnwidth}\raggedright
\passthrough{\lstinline!get\_repo\_root!}\strut
\end{minipage} & \begin{minipage}[t]{0.17\columnwidth}\raggedright
function\strut
\end{minipage} & \begin{minipage}[t]{0.17\columnwidth}\raggedright
\passthrough{\lstinline!(start)!}\strut
\end{minipage} & \begin{minipage}[t]{0.17\columnwidth}\raggedright
Heuristically find repository root by walking up from
\passthrough{\lstinline!start!}.\strut
\end{minipage}\tabularnewline
\begin{minipage}[t]{0.17\columnwidth}\raggedright
\passthrough{\lstinline!quadray!}\strut
\end{minipage} & \begin{minipage}[t]{0.17\columnwidth}\raggedright
\passthrough{\lstinline!DEFAULT\_EMBEDDING!}\strut
\end{minipage} & \begin{minipage}[t]{0.17\columnwidth}\raggedright
constant\strut
\end{minipage} & \begin{minipage}[t]{0.17\columnwidth}\raggedright
\passthrough{\lstinline!|  | | `quadray` | `Quadray` | class |!}\strut
\end{minipage} & \begin{minipage}[t]{0.17\columnwidth}\raggedright
Quadray vector with non-negative components and at least one zero
(Fuller.4D).\strut
\end{minipage}\tabularnewline
\begin{minipage}[t]{0.17\columnwidth}\raggedright
\passthrough{\lstinline!quadray!}\strut
\end{minipage} & \begin{minipage}[t]{0.17\columnwidth}\raggedright
\passthrough{\lstinline!ace\_tetravolume\_5x5!}\strut
\end{minipage} & \begin{minipage}[t]{0.17\columnwidth}\raggedright
function\strut
\end{minipage} & \begin{minipage}[t]{0.17\columnwidth}\raggedright
\passthrough{\lstinline!(p0, p1, p2, p3)!}\strut
\end{minipage} & \begin{minipage}[t]{0.17\columnwidth}\raggedright
Tom Ace 5x5 determinant in IVM units (Fuller.4D).\strut
\end{minipage}\tabularnewline
\begin{minipage}[t]{0.17\columnwidth}\raggedright
\passthrough{\lstinline!quadray!}\strut
\end{minipage} & \begin{minipage}[t]{0.17\columnwidth}\raggedright
\passthrough{\lstinline!dot!}\strut
\end{minipage} & \begin{minipage}[t]{0.17\columnwidth}\raggedright
function\strut
\end{minipage} & \begin{minipage}[t]{0.17\columnwidth}\raggedright
\passthrough{\lstinline!(q1, q2, embedding)!}\strut
\end{minipage} & \begin{minipage}[t]{0.17\columnwidth}\raggedright
Return Euclidean dot product \textless q1,q2\textgreater{} under the
given embedding.\strut
\end{minipage}\tabularnewline
\begin{minipage}[t]{0.17\columnwidth}\raggedright
\passthrough{\lstinline!quadray!}\strut
\end{minipage} & \begin{minipage}[t]{0.17\columnwidth}\raggedright
\passthrough{\lstinline!integer\_tetra\_volume!}\strut
\end{minipage} & \begin{minipage}[t]{0.17\columnwidth}\raggedright
function\strut
\end{minipage} & \begin{minipage}[t]{0.17\columnwidth}\raggedright
\passthrough{\lstinline!(p0, p1, p2, p3)!}\strut
\end{minipage} & \begin{minipage}[t]{0.17\columnwidth}\raggedright
Compute integer tetra-volume using det{[}p1-p0, p2-p0, p3-p0{]}
(Fuller.4D).\strut
\end{minipage}\tabularnewline
\begin{minipage}[t]{0.17\columnwidth}\raggedright
\passthrough{\lstinline!quadray!}\strut
\end{minipage} & \begin{minipage}[t]{0.17\columnwidth}\raggedright
\passthrough{\lstinline!magnitude!}\strut
\end{minipage} & \begin{minipage}[t]{0.17\columnwidth}\raggedright
function\strut
\end{minipage} & \begin{minipage}[t]{0.17\columnwidth}\raggedright
\passthrough{\lstinline!(q, embedding)!}\strut
\end{minipage} & \begin{minipage}[t]{0.17\columnwidth}\raggedright
Return Euclidean magnitude \textbar\textbar q\textbar\textbar{} under
the given embedding (vector norm).\strut
\end{minipage}\tabularnewline
\begin{minipage}[t]{0.17\columnwidth}\raggedright
\passthrough{\lstinline!quadray!}\strut
\end{minipage} & \begin{minipage}[t]{0.17\columnwidth}\raggedright
\passthrough{\lstinline!to\_xyz!}\strut
\end{minipage} & \begin{minipage}[t]{0.17\columnwidth}\raggedright
function\strut
\end{minipage} & \begin{minipage}[t]{0.17\columnwidth}\raggedright
\passthrough{\lstinline!(q, embedding)!}\strut
\end{minipage} & \begin{minipage}[t]{0.17\columnwidth}\raggedright
Map quadray to R\^{}3 via a 3x4 embedding matrix (Fuller.4D
-\textgreater{} Coxeter.4D slice).\strut
\end{minipage}\tabularnewline
\begin{minipage}[t]{0.17\columnwidth}\raggedright
\passthrough{\lstinline!symbolic!}\strut
\end{minipage} & \begin{minipage}[t]{0.17\columnwidth}\raggedright
\passthrough{\lstinline!cayley\_menger\_volume\_symbolic!}\strut
\end{minipage} & \begin{minipage}[t]{0.17\columnwidth}\raggedright
function\strut
\end{minipage} & \begin{minipage}[t]{0.17\columnwidth}\raggedright
\passthrough{\lstinline!(d2)!}\strut
\end{minipage} & \begin{minipage}[t]{0.17\columnwidth}\raggedright
Return symbolic Euclidean tetrahedron volume from squared
distances.\strut
\end{minipage}\tabularnewline
\begin{minipage}[t]{0.17\columnwidth}\raggedright
\passthrough{\lstinline!symbolic!}\strut
\end{minipage} & \begin{minipage}[t]{0.17\columnwidth}\raggedright
\passthrough{\lstinline!convert\_xyz\_volume\_to\_ivm\_symbolic!}\strut
\end{minipage} & \begin{minipage}[t]{0.17\columnwidth}\raggedright
function\strut
\end{minipage} & \begin{minipage}[t]{0.17\columnwidth}\raggedright
\passthrough{\lstinline!(V\_xyz)!}\strut
\end{minipage} & \begin{minipage}[t]{0.17\columnwidth}\raggedright
Convert a symbolic Euclidean volume to IVM tetravolume via S3.\strut
\end{minipage}\tabularnewline
\begin{minipage}[t]{0.17\columnwidth}\raggedright
\passthrough{\lstinline!visualize!}\strut
\end{minipage} & \begin{minipage}[t]{0.17\columnwidth}\raggedright
\passthrough{\lstinline!animate\_discrete\_path!}\strut
\end{minipage} & \begin{minipage}[t]{0.17\columnwidth}\raggedright
function\strut
\end{minipage} & \begin{minipage}[t]{0.17\columnwidth}\raggedright
\passthrough{\lstinline!(path, embedding, save)!}\strut
\end{minipage} & \begin{minipage}[t]{0.17\columnwidth}\raggedright
Animate a point moving along a discrete quadray path.\strut
\end{minipage}\tabularnewline
\begin{minipage}[t]{0.17\columnwidth}\raggedright
\passthrough{\lstinline!visualize!}\strut
\end{minipage} & \begin{minipage}[t]{0.17\columnwidth}\raggedright
\passthrough{\lstinline!animate\_simplex!}\strut
\end{minipage} & \begin{minipage}[t]{0.17\columnwidth}\raggedright
function\strut
\end{minipage} & \begin{minipage}[t]{0.17\columnwidth}\raggedright
\passthrough{\lstinline!(vertices\_list, embedding, save)!}\strut
\end{minipage} & \begin{minipage}[t]{0.17\columnwidth}\raggedright
Animate simplex evolution across iterations.\strut
\end{minipage}\tabularnewline
\begin{minipage}[t]{0.17\columnwidth}\raggedright
\passthrough{\lstinline!visualize!}\strut
\end{minipage} & \begin{minipage}[t]{0.17\columnwidth}\raggedright
\passthrough{\lstinline!plot\_ivm\_neighbors!}\strut
\end{minipage} & \begin{minipage}[t]{0.17\columnwidth}\raggedright
function\strut
\end{minipage} & \begin{minipage}[t]{0.17\columnwidth}\raggedright
\passthrough{\lstinline!(embedding, save)!}\strut
\end{minipage} & \begin{minipage}[t]{0.17\columnwidth}\raggedright
Scatter the 12 IVM neighbor points in 3D.\strut
\end{minipage}\tabularnewline
\begin{minipage}[t]{0.17\columnwidth}\raggedright
\passthrough{\lstinline!visualize!}\strut
\end{minipage} & \begin{minipage}[t]{0.17\columnwidth}\raggedright
\passthrough{\lstinline!plot\_partition\_tetrahedron!}\strut
\end{minipage} & \begin{minipage}[t]{0.17\columnwidth}\raggedright
function\strut
\end{minipage} & \begin{minipage}[t]{0.17\columnwidth}\raggedright
\passthrough{\lstinline!(mu, s, a, psi, embedding, save)!}\strut
\end{minipage} & \begin{minipage}[t]{0.17\columnwidth}\raggedright
Plot the four-fold partition as a labeled tetrahedron in 3D.\strut
\end{minipage}\tabularnewline
\begin{minipage}[t]{0.17\columnwidth}\raggedright
\passthrough{\lstinline!visualize!}\strut
\end{minipage} & \begin{minipage}[t]{0.17\columnwidth}\raggedright
\passthrough{\lstinline!plot\_simplex\_trace!}\strut
\end{minipage} & \begin{minipage}[t]{0.17\columnwidth}\raggedright
function\strut
\end{minipage} & \begin{minipage}[t]{0.17\columnwidth}\raggedright
\passthrough{\lstinline!(state, save)!}\strut
\end{minipage} & \begin{minipage}[t]{0.17\columnwidth}\raggedright
Plot per-iteration diagnostics for Nelder--Mead.\strut
\end{minipage}\tabularnewline
\bottomrule
\end{longtable}

\end{document}
