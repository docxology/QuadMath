% Options for packages loaded elsewhere
\PassOptionsToPackage{unicode}{hyperref}
\PassOptionsToPackage{hyphens}{url}
\PassOptionsToPackage{dvipsnames,svgnames*,x11names*}{xcolor}
%
\documentclass[
  10pt,
]{article}
\usepackage{lmodern}
\usepackage{setspace}
\usepackage{amssymb,amsmath}
\usepackage{ifxetex,ifluatex}
\ifnum 0\ifxetex 1\fi\ifluatex 1\fi=0 % if pdftex
  \usepackage[T1]{fontenc}
  \usepackage[utf8]{inputenc}
  \usepackage{textcomp} % provide euro and other symbols
\else % if luatex or xetex
  \usepackage{unicode-math}
  \defaultfontfeatures{Scale=MatchLowercase}
  \defaultfontfeatures[\rmfamily]{Ligatures=TeX,Scale=1}
  \setmainfont[]{DejaVu Serif}
  \setmonofont[]{DejaVu Sans Mono}
\fi
% Use upquote if available, for straight quotes in verbatim environments
\IfFileExists{upquote.sty}{\usepackage{upquote}}{}
\IfFileExists{microtype.sty}{% use microtype if available
  \usepackage[]{microtype}
  \UseMicrotypeSet[protrusion]{basicmath} % disable protrusion for tt fonts
}{}
\makeatletter
\@ifundefined{KOMAClassName}{% if non-KOMA class
  \IfFileExists{parskip.sty}{%
    \usepackage{parskip}
  }{% else
    \setlength{\parindent}{0pt}
    \setlength{\parskip}{6pt plus 2pt minus 1pt}}
}{% if KOMA class
  \KOMAoptions{parskip=half}}
\makeatother
\usepackage{xcolor}
\IfFileExists{xurl.sty}{\usepackage{xurl}}{} % add URL line breaks if available
\IfFileExists{bookmark.sty}{\usepackage{bookmark}}{\usepackage{hyperref}}
\hypersetup{
  colorlinks=true,
  linkcolor=red,
  filecolor=red,
  citecolor=red,
  urlcolor=red,
  pdfcreator={LaTeX via pandoc}}
\urlstyle{same} % disable monospaced font for URLs
\usepackage[margin=1cm,top=1cm,bottom=1cm,left=1cm,right=1cm,includeheadfoot]{geometry}
\usepackage{listings}
\newcommand{\passthrough}[1]{#1}
\lstset{defaultdialect=[5.3]Lua}
\lstset{defaultdialect=[x86masm]Assembler}
\setlength{\emergencystretch}{3em} % prevent overfull lines
\providecommand{\tightlist}{%
  \setlength{\itemsep}{0pt}\setlength{\parskip}{0pt}}
\setcounter{secnumdepth}{3}
% Enable graphics inclusion and ensure figure numbering works
\usepackage{graphicx}
\renewcommand{\figurename}{Figure}

% Configure fonts for Unicode support with fallbacks
\usepackage{newunicodechar}
\newunicodechar{⁴}{\textsuperscript{4}}
\newunicodechar{₄}{\textsubscript{4}}

% Enhanced code block styling for better contrast and readability
\usepackage{fancyvrb}
\usepackage{xcolor}
\usepackage{listings}

% Define custom colors for code blocks
\definecolor{codebg}{RGB}{245, 245, 245}      % Light gray background
\definecolor{codeborder}{RGB}{200, 200, 200}  % Medium gray border
\definecolor{codefg}{RGB}{50, 50, 50}         % Dark gray text

% Configure Verbatim environment for inline code
\DefineVerbatimEnvironment{Verbatim}{Verbatim}{%
    fontsize=\small,
    frame=single,
    framerule=0.5pt,
    framesep=3pt,
    rulecolor=\color{codeborder},
    bgcolor=\color{codebg},
    fgcolor=\color{codefg}
}

% Configure code block styling
\DefineVerbatimEnvironment{Highlighting}{Verbatim}{%
    fontsize=\footnotesize,
    frame=single,
    framerule=0.5pt,
    framesep=5pt,
    rulecolor=\color{codeborder},
    bgcolor=\color{codebg},
    fgcolor=\color{codefg}
}

% Style inline code with \texttt
\renewcommand{\texttt}[1]{%
    \colorbox{codebg}{\color{codefg}\ttfamily #1}%
}

% Configure listings package for code blocks
\lstset{
    backgroundcolor=\color{codebg},
    basicstyle=\footnotesize\ttfamily\color{codefg},
    breakatwhitespace=false,
    breaklines=true,
    captionpos=b,
    commentstyle=\color{codefg},
    deletekeywords={...},
    escapeinside={\%*}{*)},
    extendedchars=true,
    frame=single,
    framerule=0.5pt,
    framesep=5pt,
    keepspaces=true,
    keywordstyle=\color{codefg},
    language=Python,
    morekeywords={*,...},
    numbers=left,
    numbersep=5pt,
    numberstyle=\tiny\color{codefg},
    rulecolor=\color{codeborder},
    showspaces=false,
    showstringspaces=false,
    showtabs=false,
    stepnumber=1,
    stringstyle=\color{codefg},
    tabsize=2,
    title=\lstname
}

% Override any Pandoc default lstset configurations
\AtBeginDocument{
    \lstset{
        backgroundcolor=\color{codebg},
        basicstyle=\footnotesize\ttfamily\color{codefg},
        frame=single,
        framerule=0.5pt,
        framesep=5pt,
        rulecolor=\color{codeborder},
        numbers=left,
        numbersep=5pt,
        numberstyle=\tiny\color{codefg}
    }
}

% Configure hyperref colors consistently
\AtBeginDocument{
% Override pandoc's hidelinks setting with consistent options
\hypersetup{
    colorlinks=true,
    allcolors=red,
    linkcolor=red,
    urlcolor=red,
    citecolor=red,
    filecolor=red,
    menucolor=red,
    linktoc=all
}
}

% Simple page break support for document structure

\title{Resources}
\author{Daniel Ari Friedman\\ ORCID: 0000-0001-6232-9096\\ Email: daniel@activeinference.institute}
\date{August 16, 2025}

\begin{document}
\maketitle

{
\hypersetup{linkcolor=black}
\setcounter{tocdepth}{3}
\tableofcontents
}
\setstretch{1.0}
\hypertarget{resources}{%
\section{Resources}\label{resources}}

This section provides comprehensive resources for learning about and
working with Quadrays, synergetics, and the computational methods
discussed in this manuscript.

\hypertarget{core-concepts-and-background}{%
\subsection{Core Concepts and
Background}\label{core-concepts-and-background}}

\hypertarget{information-geometry-and-optimization}{%
\subsubsection{Information Geometry and
Optimization}\label{information-geometry-and-optimization}}

\begin{itemize}
\tightlist
\item
  \textbf{Fisher information}:
  \href{https://en.wikipedia.org/wiki/Fisher_information}{Fisher
  information (reference)} --- see also Eq. \eqref{eq:supp_fim} in the
  equations appendix
\item
  \textbf{Natural gradient}:
  \href{https://en.wikipedia.org/wiki/Natural_gradient}{Natural gradient
  (reference)} --- see also Eq. \eqref{eq:supp_natgrad} in the equations
  appendix
\end{itemize}

\hypertarget{active-inference-and-free-energy}{%
\subsubsection{Active Inference and Free
Energy}\label{active-inference-and-free-energy}}

\begin{itemize}
\tightlist
\item
  \textbf{Active Inference Institute}:
  \href{https://welcome.activeinference.institute/}{Welcome to Active
  Inference Institute}
\item
  \textbf{Comprehensive review}:
  \href{https://discovery.ucl.ac.uk/id/eprint/10176959/1/1-s2.0-S1571064523001094-main.pdf}{Active
  Inference --- recent review (UCL Discovery, 2023)}
\end{itemize}

\hypertarget{mathematical-foundations}{%
\subsubsection{Mathematical
Foundations}\label{mathematical-foundations}}

\begin{itemize}
\tightlist
\item
  \textbf{Tetrahedron volume formulas}: length-based
  \href{https://en.wikipedia.org/wiki/Cayley\%E2\%80\%93Menger_determinant}{Cayley--Menger
  determinant} and determinant-based expressions on vertex coordinates
  (see
  \href{https://en.wikipedia.org/wiki/Tetrahedron\#Volume}{Tetrahedron
  -- volume})
\item
  \textbf{Exact determinants}:
  \href{https://en.wikipedia.org/wiki/Bareiss_algorithm}{Bareiss
  algorithm}, used in our integer tetravolume implementations
\item
  \textbf{Optimization baseline}: the
  \href{https://en.wikipedia.org/wiki/Nelder\%E2\%80\%93Mead_method}{Nelder--Mead
  method}, adapted here to the Quadray lattice
\end{itemize}

\hypertarget{quadrays-and-synergetics-core-starting-points}{%
\subsection{Quadrays and Synergetics (Core Starting
Points)}\label{quadrays-and-synergetics-core-starting-points}}

\hypertarget{introductory-materials}{%
\subsubsection{Introductory Materials}\label{introductory-materials}}

\begin{itemize}
\tightlist
\item
  \textbf{Quadray coordinates (intro and conversions)}:
  \href{https://www.grunch.net/synergetics/quadintro.html}{Urner --
  Quadray intro},
  \href{https://www.grunch.net/synergetics/quadxyz.html}{Urner --
  Quadrays and XYZ}
\item
  \textbf{Quadrays and the Philosophy of Mathematics}:
  \href{https://www.grunch.net/synergetics/quadphil.html}{Urner --
  Quadrays and the Philosophy of Mathematics}
\item
  \textbf{Synergetics background and IVM}:
  \href{https://en.wikipedia.org/wiki/Synergetics_(Fuller)}{Synergetics
  (Fuller, overview)}
\item
  \textbf{Quadray coordinates overview}:
  \href{https://en.wikipedia.org/wiki/Quadray_coordinates}{Quadray
  coordinates (reference)}
\end{itemize}

\hypertarget{historical-and-background-materials}{%
\subsubsection{Historical and Background
Materials}\label{historical-and-background-materials}}

\begin{itemize}
\tightlist
\item
  \textbf{RW Gray projects --- Synergetics text}:
  \href{http://www.rwgrayprojects.com/synergetics/s00/p0000.html}{rwgrayprojects.com
  (synergetics)}
\item
  \textbf{Fuller FAQ}:
  \href{https://www.cjfearnley.com/fuller-faq.pdf}{C. J. Fearnley's
  Fuller FAQ}
\item
  \textbf{Synergetics resource list}:
  \href{https://www.cjfearnley.com/fuller-faq-2.html}{C. J. Fearnley's
  resource page}
\item
  \textbf{Wikieducator}:
  \href{https://wikieducator.org/Synergetics}{Synergetics hub}
\item
  \textbf{Quadray animation}:
  \href{https://commons.wikimedia.org/wiki/File:Quadray.gif}{Quadray.gif
  (Wikimedia Commons)}
\item
  \textbf{Fuller Institute}:
  \href{https://www.bfi.org/about-fuller/big-ideas/synergetics/}{BFI ---
  Big Ideas: Synergetics}
\end{itemize}

\hypertarget{dsolutions-ecosystem-comprehensive-computational-framework}{%
\subsection{4dsolutions Ecosystem: Comprehensive Computational
Framework}\label{dsolutions-ecosystem-comprehensive-computational-framework}}

The \href{https://github.com/4dsolutions}{4dsolutions organization}
provides the most extensive computational framework for Quadrays and
synergetic geometry, spanning 29+ repositories with implementations
across multiple programming languages.

\hypertarget{core-computational-modules}{%
\subsubsection{Core Computational
Modules}\label{core-computational-modules}}

\hypertarget{primary-python-libraries}{%
\paragraph{Primary Python Libraries}\label{primary-python-libraries}}

\begin{itemize}
\tightlist
\item
  \textbf{Math for Wisdom (m4w)}:
  \href{https://github.com/4dsolutions/m4w}{m4w (repo)}

  \begin{itemize}
  \tightlist
  \item
    \textbf{Quadray vectors and conversions}:
    \href{https://github.com/4dsolutions/m4w/blob/main/qrays.py}{\passthrough{\lstinline!qrays.py!}
    (Qvector, SymPy-aware)}
  \item
    \textbf{Synergetic tetravolumes and modules}:
    \href{https://github.com/4dsolutions/m4w/blob/main/tetravolume.py}{\passthrough{\lstinline!tetravolume.py!}
    with PdF-CM vs native IVM and BEAST algorithms}
  \end{itemize}
\end{itemize}

\hypertarget{cross-language-validation}{%
\paragraph{Cross-Language Validation}\label{cross-language-validation}}

\begin{itemize}
\tightlist
\item
  \textbf{Rust implementation}:
  \href{https://github.com/4dsolutions/rusty_rays}{rusty\_rays}
  (performance-oriented)

  \begin{itemize}
  \tightlist
  \item
    Sources:
    \href{https://github.com/4dsolutions/rusty_rays/blob/master/src/lib.rs}{Rust
    library implementation},
    \href{https://github.com/4dsolutions/rusty_rays/blob/master/src/main.rs}{Rust
    command-line interface}
  \end{itemize}
\item
  \textbf{Clojure implementation}:
  \href{https://github.com/4dsolutions/synmods}{synmods} (functional
  paradigm)

  \begin{itemize}
  \tightlist
  \item
    Sources:
    \href{https://github.com/4dsolutions/synmods/blob/master/qrays.clj}{\passthrough{\lstinline!qrays.clj!}},
    \href{https://github.com/4dsolutions/synmods/blob/master/ramping_up.clj}{\passthrough{\lstinline!ramping\_up.clj!}}
  \end{itemize}
\end{itemize}

\hypertarget{primary-hub-school_of_tomorrow-python-notebooks}{%
\subsubsection{Primary Hub: School\_of\_Tomorrow (Python +
Notebooks)}\label{primary-hub-school_of_tomorrow-python-notebooks}}

\textbf{Repository}:
\href{https://github.com/4dsolutions/School_of_Tomorrow}{School\_of\_Tomorrow}

\hypertarget{core-modules}{%
\paragraph{Core Modules}\label{core-modules}}

\begin{itemize}
\tightlist
\item
  \textbf{\passthrough{\lstinline!qrays.py!}}: Quadray implementation
  with normalization, conversions, and vector ops
  (\href{https://github.com/4dsolutions/School_of_Tomorrow/blob/master/qrays.py}{source})
\item
  \textbf{\passthrough{\lstinline!quadcraft.py!}}: POV-Ray scenes for
  CCP/IVM arrangements, animations, and tutorials
  (\href{https://github.com/4dsolutions/School_of_Tomorrow/blob/master/quadcraft.py}{source})
\item
  \textbf{\passthrough{\lstinline!flextegrity.py!}}: Polyhedron
  framework, concentric hierarchy, POV-Ray export
  (\href{https://github.com/4dsolutions/School_of_Tomorrow/blob/master/flextegrity.py}{source})
\item
  \textbf{Additional modules}: \passthrough{\lstinline!polyhedra.py!},
  \passthrough{\lstinline!identities.py!},
  \passthrough{\lstinline!smod\_play.py!} (synergetic modules)
\end{itemize}

\hypertarget{key-notebooks}{%
\paragraph{Key Notebooks}\label{key-notebooks}}

\begin{itemize}
\tightlist
\item
  \textbf{\passthrough{\lstinline!Qvolume.ipynb!}}: Tom Ace 5×5
  determinant with random-walk demonstrations
  (\href{https://github.com/4dsolutions/School_of_Tomorrow/blob/master/Qvolume.ipynb}{source})
\item
  \textbf{\passthrough{\lstinline!VolumeTalk.ipynb!}}: Comparative
  analysis of bridging vs native tetravolume formulations
  (\href{https://github.com/4dsolutions/School_of_Tomorrow/blob/master/VolumeTalk.ipynb}{source})
\item
  \textbf{\passthrough{\lstinline!QuadCraft\_Project.ipynb!}}: 1,255
  lines of interactive CCP navigation and visualization tutorials
  (\href{https://github.com/4dsolutions/School_of_Tomorrow/blob/master/QuadCraft_Project.ipynb}{source})
\item
  \textbf{Additional notebooks}:
  \passthrough{\lstinline!TetraBook.ipynb!},
  \passthrough{\lstinline!CascadianSynergetics.ipynb!},
  \passthrough{\lstinline!Rendering\_IVM.ipynb!},
  \passthrough{\lstinline!SphereVolumes.ipynb!} (visual and curricular
  materials)
\end{itemize}

\hypertarget{additional-repositories}{%
\subsubsection{Additional Repositories}\label{additional-repositories}}

\hypertarget{tetravolumes-algorithms-and-pedagogy}{%
\paragraph{Tetravolumes (Algorithms and
Pedagogy)}\label{tetravolumes-algorithms-and-pedagogy}}

\begin{itemize}
\tightlist
\item
  \textbf{Repository}:
  \href{https://github.com/4dsolutions/tetravolumes}{tetravolumes}
\item
  \textbf{Code}:
  \href{https://github.com/4dsolutions/tetravolumes/blob/master/tetravolume.py}{\passthrough{\lstinline!tetravolume.py!}}
\item
  \textbf{Notebooks}:
  \href{https://raw.githubusercontent.com/4dsolutions/tetravolumes/refs/heads/master/Atoms\%20R\%20Us.ipynb}{Atoms
  R Us.ipynb},
  \href{https://raw.githubusercontent.com/4dsolutions/tetravolumes/refs/heads/master/Computing\%20Volumes.ipynb}{Computing
  Volumes.ipynb}
\end{itemize}

\hypertarget{visualization-and-rendering}{%
\paragraph{Visualization and
Rendering}\label{visualization-and-rendering}}

\begin{itemize}
\tightlist
\item
  \textbf{BookCovers}: VPython for interactive educational animations
  (\href{https://github.com/4dsolutions/BookCovers}{repo})

  \begin{itemize}
  \tightlist
  \item
    Examples:
    \href{https://github.com/4dsolutions/BookCovers/blob/master/bookdemo.py}{\passthrough{\lstinline!bookdemo.py!}},
    \href{https://github.com/4dsolutions/BookCovers/blob/master/stickworks.py}{\passthrough{\lstinline!stickworks.py!}},
    \href{https://github.com/4dsolutions/BookCovers/blob/master/tetravolumes.py}{\passthrough{\lstinline!tetravolumes.py!}}
  \end{itemize}
\end{itemize}

\hypertarget{educational-framework-and-curricula}{%
\subsubsection{Educational Framework and
Curricula}\label{educational-framework-and-curricula}}

\hypertarget{oregon-curriculum-network-ocn}{%
\paragraph{Oregon Curriculum Network
(OCN)}\label{oregon-curriculum-network-ocn}}

\begin{itemize}
\tightlist
\item
  \textbf{OCN portal}: \href{http://www.4dsolutions.net/ocn/}{OCN
  portal}
\item
  \textbf{Python for Everyone}:
  \href{http://www.4dsolutions.net/ocn/pymath.html}{pymath page}
\end{itemize}

\hypertarget{historical-documentation}{%
\paragraph{Historical Documentation}\label{historical-documentation}}

\begin{itemize}
\tightlist
\item
  \textbf{Python5 notebooks}:
  \href{https://raw.githubusercontent.com/4dsolutions/Python5/master/Polyhedrons\%20101.ipynb}{Polyhedrons
  101.ipynb}
\item
  \textbf{Historical variants}: \passthrough{\lstinline!qrays.py!} also
  appears in
  \href{https://github.com/4dsolutions/Python5/blob/master/qrays.py}{Python5
  (archive)}
\item
  \textbf{Python edu-sig archives}:
  \href{https://mail.python.org/pipermail/edu-sig/2000-May/000498.html}{Python
  edu-sig archives} tracing 25+ years of development
\end{itemize}

\hypertarget{media-and-publications}{%
\subsubsection{Media and Publications}\label{media-and-publications}}

\begin{itemize}
\tightlist
\item
  \textbf{YouTube demonstrations}:
  \href{https://www.youtube.com/watch?v=g14mu4uWD4E}{Synergetics talk
  1}, \href{https://www.youtube.com/watch?v=i9oij02oje0}{Synergetics
  talk 2},
  \href{https://www.youtube.com/watch?v=D0M1h_gjA_w}{Additional}
\item
  \textbf{Academia profile}:
  \href{https://princeton.academia.edu/kirbyurner}{Kirby Urner at
  Academia.edu}
\end{itemize}

\hypertarget{community-discussions-and-collaborative-platforms}{%
\subsection{Community Discussions and Collaborative
Platforms}\label{community-discussions-and-collaborative-platforms}}

\hypertarget{active-platforms}{%
\subsubsection{Active Platforms}\label{active-platforms}}

\begin{itemize}
\tightlist
\item
  \textbf{Math4Wisdom Knowledge Engineering}:
  \href{https://coda.io/d/_d0SvdI3KSto/Knowledge-Engineering_suxu39sp}{Collaborative
  platform} with various art, resources, and cross-reference materials
\item
  \textbf{synergeo discussion archive}:
  \href{https://groups.io/g/synergeo/topics}{Groups.io platform} with
  ongoing community discussions and technical exchanges
\end{itemize}

\hypertarget{historical-archives}{%
\subsubsection{Historical Archives}\label{historical-archives}}

\begin{itemize}
\tightlist
\item
  \textbf{GeodesicHelp threads}:
  \href{https://groups.google.com/g/GeodesicHelp/}{GeodesicHelp
  computations archive (Google Groups)} documenting computational
  approaches and problem-solving techniques
\end{itemize}

\hypertarget{related-projects-and-applications}{%
\subsection{Related Projects and
Applications}\label{related-projects-and-applications}}

\hypertarget{tetrahedral-voxel-engines}{%
\subsubsection{Tetrahedral Voxel
Engines}\label{tetrahedral-voxel-engines}}

\begin{itemize}
\tightlist
\item
  \textbf{QuadCraft}:
  \href{https://github.com/docxology/quadcraft/}{Tetrahedral voxel
  engine using Quadrays}
\end{itemize}

\hypertarget{academic-publications}{%
\subsubsection{Academic Publications}\label{academic-publications}}

\begin{itemize}
\tightlist
\item
  \textbf{Flextegrity}:
  \href{https://www.academia.edu/44531954/Generating_the_Flextegrity_Lattice}{Generating
  the Flextegrity Lattice (academia.edu)}
\end{itemize}

\hypertarget{context-and-integration}{%
\subsubsection{Context and Integration}\label{context-and-integration}}

These materials popularize the IVM/CCP/FCC framing of space, integer
tetravolumes, and projective Quadray normalization. They inform the
methods in this paper and complement the \passthrough{\lstinline!src/!}
implementations (see \passthrough{\lstinline!quadray.py!},
\passthrough{\lstinline!cayley\_menger.py!},
\passthrough{\lstinline!linalg\_utils.py!}).

The ecosystem provides extensive validation, pedagogical context, and
practical implementations that complement and extend the methods
developed in this manuscript. Cross-language implementations serve as
independent verification of algorithmic correctness while educational
materials demonstrate practical applications across diverse
computational environments.

\end{document}
