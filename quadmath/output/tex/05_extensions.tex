% Options for packages loaded elsewhere
\PassOptionsToPackage{unicode}{hyperref}
\PassOptionsToPackage{hyphens}{url}
\PassOptionsToPackage{dvipsnames,svgnames*,x11names*}{xcolor}
%
\documentclass[
  10pt,
]{article}
\usepackage{lmodern}
\usepackage{setspace}
\usepackage{amssymb,amsmath}
\usepackage{ifxetex,ifluatex}
\ifnum 0\ifxetex 1\fi\ifluatex 1\fi=0 % if pdftex
  \usepackage[T1]{fontenc}
  \usepackage[utf8]{inputenc}
  \usepackage{textcomp} % provide euro and other symbols
\else % if luatex or xetex
  \usepackage{unicode-math}
  \defaultfontfeatures{Scale=MatchLowercase}
  \defaultfontfeatures[\rmfamily]{Ligatures=TeX,Scale=1}
  \setmainfont[]{DejaVu Serif}
  \setmonofont[]{DejaVu Sans Mono}
\fi
% Use upquote if available, for straight quotes in verbatim environments
\IfFileExists{upquote.sty}{\usepackage{upquote}}{}
\IfFileExists{microtype.sty}{% use microtype if available
  \usepackage[]{microtype}
  \UseMicrotypeSet[protrusion]{basicmath} % disable protrusion for tt fonts
}{}
\makeatletter
\@ifundefined{KOMAClassName}{% if non-KOMA class
  \IfFileExists{parskip.sty}{%
    \usepackage{parskip}
  }{% else
    \setlength{\parindent}{0pt}
    \setlength{\parskip}{6pt plus 2pt minus 1pt}}
}{% if KOMA class
  \KOMAoptions{parskip=half}}
\makeatother
\usepackage{xcolor}
\IfFileExists{xurl.sty}{\usepackage{xurl}}{} % add URL line breaks if available
\IfFileExists{bookmark.sty}{\usepackage{bookmark}}{\usepackage{hyperref}}
\hypersetup{
  colorlinks=true,
  linkcolor=red,
  filecolor=red,
  citecolor=red,
  urlcolor=red,
  pdfcreator={LaTeX via pandoc}}
\urlstyle{same} % disable monospaced font for URLs
\usepackage[margin=0.7cm,top=0.7cm,bottom=0.7cm,left=0.7cm,right=0.7cm,includeheadfoot]{geometry}
\setlength{\emergencystretch}{3em} % prevent overfull lines
\providecommand{\tightlist}{%
  \setlength{\itemsep}{0pt}\setlength{\parskip}{0pt}}
\setcounter{secnumdepth}{3}
% Enable graphics inclusion and ensure figure numbering works
\usepackage{graphicx}
\renewcommand{\figurename}{Figure}

% Configure fonts for Unicode support with fallbacks
\usepackage{newunicodechar}
\newunicodechar{⁴}{\textsuperscript{4}}
\newunicodechar{₄}{\textsubscript{4}}

% Configure hyperref colors consistently
\AtBeginDocument{
% Override pandoc's hidelinks setting with consistent options
\hypersetup{
    colorlinks=true,
    allcolors=red,
    linkcolor=red,
    urlcolor=red,
    citecolor=red,
    filecolor=red,
    menucolor=red,
    linktoc=all
}
}

\title{Extensions of 4D and Quadrays}
\author{Daniel Ari Friedman\\ ORCID: 0000-0001-6232-9096\\ Email: daniel@activeinference.institute}
\date{August 14, 2025}

\begin{document}
\maketitle

{
\hypersetup{linkcolor=red}
\setcounter{tocdepth}{3}
\tableofcontents
}
\setstretch{1.0}
\hypertarget{extensions-of-4d-and-quadrays}{%
\section{Extensions of 4D and
Quadrays}\label{extensions-of-4d-and-quadrays}}

Here we review some extensions of the Quadray 4D framework, including
multi-objective optimization, machine learning, active inference,
complex systems, pedagogy, and implementations, with an emphasis on
cognitive security.

\hypertarget{multi-objective-optimization}{%
\subsection{Multi-Objective
Optimization}\label{multi-objective-optimization}}

\begin{itemize}
\tightlist
\item
  Simplex faces encode trade-offs; integer volume measures solution
  diversity.
\item
  Pareto front exploration via tetrahedral traversal.
\end{itemize}

\hypertarget{machine-learning-and-robustness}{%
\subsection{Machine Learning and
Robustness}\label{machine-learning-and-robustness}}

\begin{itemize}
\tightlist
\item
  \textbf{Geometric regularization}: Quadray-constrained
  weights/topologies yield structural priors and improved stability.
\item
  \textbf{Adversarial robustness}: Discrete lattice projection reduces
  vulnerability to gradient-based adversarial perturbations by limiting
  directions.
\item
  \textbf{Ensembles}: Tetrahedral vertex voting and consensus improve
  robustness.
\end{itemize}

References: see
\href{https://en.wikipedia.org/wiki/Fisher_information}{Fisher
information},
\href{https://en.wikipedia.org/wiki/Natural_gradient}{Natural gradient},
and quadray conversion notes by Urner for embedding choices.

\hypertarget{active-inference-and-free-energy}{%
\subsection{Active Inference and Free
Energy}\label{active-inference-and-free-energy}}

\begin{itemize}
\tightlist
\item
  Free energy
  \(\mathcal{F} = -\log P(o\mid s) + \mathrm{KL}[Q(s)\,\|\,P(s)]\) (see
  Eq. \eqref{eq:supp_free_energy}); background:
  \href{https://en.wikipedia.org/wiki/Free_energy_principle}{Free energy
  principle} and overviews connecting to predictive coding and control.
\item
  Belief updates follow steepest descent in Fisher geometry using the
  natural gradient (see Eq. \eqref{eq:supp_natgrad}); quadray
  constraints improve stability/interpretability.
\item
  Links to metabolic efficiency and biologically plausible computation.
\end{itemize}

\hypertarget{complex-systems-and-collective-intelligence}{%
\subsection{Complex Systems and Collective
Intelligence}\label{complex-systems-and-collective-intelligence}}

\begin{itemize}
\tightlist
\item
  Tetrahedral interaction patterns support distributed consensus and
  emergent behavior.
\item
  Resource allocation and network flows benefit from geometric
  constraints.
\item
  \textbf{Cognitive security}: Applying cognitive security can safeguard
  distributed consensus mechanisms from manipulation, preserving the
  reliability of emergent behaviors in complex systems. Incorporating
  cognitive security measures can protect the integrity of belief
  updates and decision-making processes, ensuring that actions are based
  on accurate and unmanipulated information.
\end{itemize}

\hypertarget{quadrays-synergetics-fuller.4d-and-william-blake}{%
\subsection{Quadrays, Synergetics (Fuller.4D), and William
Blake}\label{quadrays-synergetics-fuller.4d-and-william-blake}}

\begin{itemize}
\tightlist
\item
  Quadrays (tetrahedral coordinates) instantiate Fuller's Synergetics
  emphasis on the tetrahedron as a structural primitive; in this
  manuscript's terminology this corresponds to Fuller.4D. Tetrahedral
  frames support part--whole reasoning and efficient decompositions used
  throughout.
\item
  William Blake's ``fourfold vision'' (single, twofold, threefold,
  fourfold) provides a historical metaphor for multiscale perception and
  inference. Read through Fisher geometry and natural gradient dynamics,
  it parallels multilayer predictive processing and counterfactual
  simulation. For background, see a concise overview of Blake's
  visionary psycho‑topographies in British Art Studies
  (\href{https://www.britishartstudies.ac.uk/index/article-index/visionary-sense-of-london/article-category/cover-collaboration}{visionary
  art analysis}) and the Active Inference Institute's MathArt Stream \#8
  (\href{https://zenodo.org/records/13711302}{Active Inference \&
  Blake}).
\item
  Juxtaposing Blake and Fuller foregrounds ``comprehensivity'': holistic
  design and sensemaking via geometric primitives. Context:
  (\href{https://zenodo.org/records/7519132}{Fuller \& Blake: Lives in
  Juxtaposition}) and pedagogical antecedents in experimental design
  education at Black Mountain College
  (\href{https://commons.princeton.edu/eng574-s23/wp-content/uploads/sites/348/2023/03/Diaz-The-Experimenters-Chance-and-Design-at-Black-Mountain-College.pdf}{Diaz,
  Chance and Design at Black Mountain College -- PDF}).
\item
  Implications for Quadray practice: four‑facet summaries of
  models/trajectories, tetrahedral consensus in ensembles, and
  stigmergic annotation patterns for cognitive security and distributed
  sensemaking.
\end{itemize}

\hypertarget{pedagogy-and-implementations}{%
\subsection{Pedagogy and
Implementations}\label{pedagogy-and-implementations}}

Kirby Urner's comprehensive
\href{https://github.com/4dsolutions}{4dsolutions ecosystem} provides
extensive educational resources and cross-platform implementations for
Quadray computation and visualization:

\hypertarget{educational-framework-and-curricula}{%
\subsubsection{Educational Framework and
Curricula}\label{educational-framework-and-curricula}}

\begin{itemize}
\tightlist
\item
  \textbf{Oregon Curriculum Network (OCN)}:
  \href{http://www.4dsolutions.net/ocn/}{OCN portal} and
  \href{http://www.4dsolutions.net/ocn/pymath.html}{Python for Everyone}
  integrate Quadrays with progressive mathematical education
\item
  \textbf{School of Tomorrow}:
  \href{https://github.com/4dsolutions/School_of_Tomorrow}{Repository}
  with comprehensive notebooks and modular teaching materials including:

  \begin{itemize}
  \tightlist
  \item
    \href{https://github.com/4dsolutions/School_of_Tomorrow/blob/master/QuadCraft_Project.ipynb}{\texttt{QuadCraft\_Project.ipynb}}:
    1,255 lines of interactive CCP navigation with QWERTY keyboard
    mapping to 12 IVM directions
  \item
    \href{https://github.com/4dsolutions/School_of_Tomorrow/blob/master/TetraBook.ipynb}{\texttt{TetraBook.ipynb}},
    \href{https://github.com/4dsolutions/School_of_Tomorrow/blob/master/CascadianSynergetics.ipynb}{\texttt{CascadianSynergetics.ipynb}}:
    Regional curriculum integration
  \item
    \href{https://github.com/4dsolutions/School_of_Tomorrow/blob/master/Rendering_IVM.ipynb}{\texttt{Rendering\_IVM.ipynb}}:
    3D visualization techniques
  \end{itemize}
\end{itemize}

\hypertarget{cross-language-implementation-portfolio}{%
\subsubsection{Cross-Language Implementation
Portfolio}\label{cross-language-implementation-portfolio}}

\begin{itemize}
\tightlist
\item
  \textbf{Python (primary)}:
  \href{https://github.com/4dsolutions/m4w/blob/main/qrays.py}{\texttt{qrays.py}}
  with SymPy integration,
  \href{https://github.com/4dsolutions/m4w/blob/main/tetravolume.py}{\texttt{tetravolume.py}}
  with multiple algorithms
\item
  \textbf{Rust (performance)}:
  \href{https://github.com/4dsolutions/rusty_rays}{\texttt{rusty\_rays}}
  for computational geometry optimization
\item
  \textbf{Clojure (functional)}:
  \href{https://github.com/4dsolutions/synmods}{\texttt{synmods}} with
  protocol-based design patterns
\item
  \textbf{POV-Ray (rendering)}:
  \href{https://github.com/4dsolutions/School_of_Tomorrow/blob/master/quadcraft.py}{\texttt{quadcraft.py}}
  with 15 test functions and automated scene generation
\item
  \textbf{VPython (interactive)}:
  \href{https://github.com/4dsolutions/BookCovers}{\texttt{BookCovers}}
  for real-time educational animations
\end{itemize}

\hypertarget{historical-context-and-evolution}{%
\subsubsection{Historical Context and
Evolution}\label{historical-context-and-evolution}}

\begin{itemize}
\tightlist
\item
  \textbf{Early innovations}:
  \href{https://mail.python.org/pipermail/edu-sig/2000-May/000498.html}{Python
  edu-sig post (May 2000)} documenting original 4D Turtle
  implementations
\item
  \textbf{Foundational materials}:
  \href{https://www.grunch.net/synergetics/quadintro.html}{Urner --
  Quadray intro} and
  \href{https://www.grunch.net/synergetics/quadxyz.html}{Quadrays and
  XYZ} conversion notes
\item
  \textbf{Community development}: Evolution through
  \href{https://coda.io/@daniel-ari-friedman/math4wisdom/ivm-xyz-40}{Math4Wisdom}
  collaboration and \href{https://groups.io/g/synergeo/topics}{synergeo}
  discussions
\end{itemize}

\hypertarget{higher-dimensions-and-decompositions}{%
\subsection{Higher Dimensions and
Decompositions}\label{higher-dimensions-and-decompositions}}

\begin{itemize}
\tightlist
\item
  Decompose higher-dimensional simplexes into tetrahedra; sum integer
  volumes to maintain quantization.
\item
  Tessellations support parallel/distributed implementations.
\end{itemize}

\hypertarget{limitations-and-future-work}{%
\subsection{Limitations and Future
Work}\label{limitations-and-future-work}}

\begin{itemize}
\tightlist
\item
  Benchmark breadth: extend beyond convex/quadratic toys to real tasks
  (registration, robust regression, control) with ablations.
\item
  Distance sensitivity: compare embeddings and their effect on optimizer
  trajectories; document recommended defaults.
\item
  Hybrid schemes: study schedules that interleave continuous proposals
  with lattice projection.
\end{itemize}

\end{document}
