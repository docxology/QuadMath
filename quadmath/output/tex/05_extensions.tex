% Options for packages loaded elsewhere
\PassOptionsToPackage{unicode}{hyperref}
\PassOptionsToPackage{hyphens}{url}
\PassOptionsToPackage{dvipsnames,svgnames*,x11names*}{xcolor}
%
\documentclass[
  10pt,
]{article}
\usepackage{lmodern}
\usepackage{setspace}
\usepackage{amssymb,amsmath}
\usepackage{ifxetex,ifluatex}
\ifnum 0\ifxetex 1\fi\ifluatex 1\fi=0 % if pdftex
  \usepackage[T1]{fontenc}
  \usepackage[utf8]{inputenc}
  \usepackage{textcomp} % provide euro and other symbols
\else % if luatex or xetex
  \usepackage{unicode-math}
  \defaultfontfeatures{Scale=MatchLowercase}
  \defaultfontfeatures[\rmfamily]{Ligatures=TeX,Scale=1}
  \setmainfont[]{DejaVu Serif}
  \setmonofont[]{DejaVu Sans Mono}
\fi
% Use upquote if available, for straight quotes in verbatim environments
\IfFileExists{upquote.sty}{\usepackage{upquote}}{}
\IfFileExists{microtype.sty}{% use microtype if available
  \usepackage[]{microtype}
  \UseMicrotypeSet[protrusion]{basicmath} % disable protrusion for tt fonts
}{}
\makeatletter
\@ifundefined{KOMAClassName}{% if non-KOMA class
  \IfFileExists{parskip.sty}{%
    \usepackage{parskip}
  }{% else
    \setlength{\parindent}{0pt}
    \setlength{\parskip}{6pt plus 2pt minus 1pt}}
}{% if KOMA class
  \KOMAoptions{parskip=half}}
\makeatother
\usepackage{xcolor}
\IfFileExists{xurl.sty}{\usepackage{xurl}}{} % add URL line breaks if available
\IfFileExists{bookmark.sty}{\usepackage{bookmark}}{\usepackage{hyperref}}
\hypersetup{
  colorlinks=true,
  linkcolor=red,
  filecolor=red,
  citecolor=red,
  urlcolor=red,
  pdfcreator={LaTeX via pandoc}}
\urlstyle{same} % disable monospaced font for URLs
\usepackage[margin=1cm,top=1cm,bottom=1cm,left=1cm,right=1cm,includeheadfoot]{geometry}
\usepackage{listings}
\newcommand{\passthrough}[1]{#1}
\lstset{defaultdialect=[5.3]Lua}
\lstset{defaultdialect=[x86masm]Assembler}
\setlength{\emergencystretch}{3em} % prevent overfull lines
\providecommand{\tightlist}{%
  \setlength{\itemsep}{0pt}\setlength{\parskip}{0pt}}
\setcounter{secnumdepth}{3}
% Enable graphics inclusion and ensure figure numbering works
\usepackage{graphicx}
\renewcommand{\figurename}{Figure}

% Configure fonts for Unicode support with fallbacks
\usepackage{newunicodechar}
\newunicodechar{⁴}{\textsuperscript{4}}
\newunicodechar{₄}{\textsubscript{4}}

% Enhanced code block styling for better contrast and readability
\usepackage{fancyvrb}
\usepackage{xcolor}
\usepackage{listings}

% Define custom colors for code blocks
\definecolor{codebg}{RGB}{245, 245, 245}      % Light gray background
\definecolor{codeborder}{RGB}{200, 200, 200}  % Medium gray border
\definecolor{codefg}{RGB}{50, 50, 50}         % Dark gray text

% Configure Verbatim environment for inline code
\DefineVerbatimEnvironment{Verbatim}{Verbatim}{%
    fontsize=\small,
    frame=single,
    framerule=0.5pt,
    framesep=3pt,
    rulecolor=\color{codeborder},
    bgcolor=\color{codebg},
    fgcolor=\color{codefg}
}

% Configure code block styling
\DefineVerbatimEnvironment{Highlighting}{Verbatim}{%
    fontsize=\footnotesize,
    frame=single,
    framerule=0.5pt,
    framesep=5pt,
    rulecolor=\color{codeborder},
    bgcolor=\color{codebg},
    fgcolor=\color{codefg}
}

% Style inline code with \texttt
\renewcommand{\texttt}[1]{%
    \colorbox{codebg}{\color{codefg}\ttfamily #1}%
}

% Configure listings package for code blocks
\lstset{
    backgroundcolor=\color{codebg},
    basicstyle=\footnotesize\ttfamily\color{codefg},
    breakatwhitespace=false,
    breaklines=true,
    captionpos=b,
    commentstyle=\color{codefg},
    deletekeywords={...},
    escapeinside={\%*}{*)},
    extendedchars=true,
    frame=single,
    framerule=0.5pt,
    framesep=5pt,
    keepspaces=true,
    keywordstyle=\color{codefg},
    language=Python,
    morekeywords={*,...},
    numbers=left,
    numbersep=5pt,
    numberstyle=\tiny\color{codefg},
    rulecolor=\color{codeborder},
    showspaces=false,
    showstringspaces=false,
    showtabs=false,
    stepnumber=1,
    stringstyle=\color{codefg},
    tabsize=2,
    title=\lstname
}

% Override any Pandoc default lstset configurations
\AtBeginDocument{
    \lstset{
        backgroundcolor=\color{codebg},
        basicstyle=\footnotesize\ttfamily\color{codefg},
        frame=single,
        framerule=0.5pt,
        framesep=5pt,
        rulecolor=\color{codeborder},
        numbers=left,
        numbersep=5pt,
        numberstyle=\tiny\color{codefg}
    }
}

% Configure hyperref colors consistently
\AtBeginDocument{
% Override pandoc's hidelinks setting with consistent options
\hypersetup{
    colorlinks=true,
    allcolors=red,
    linkcolor=red,
    urlcolor=red,
    citecolor=red,
    filecolor=red,
    menucolor=red,
    linktoc=all
}
}

% Simple page break support for document structure

\title{Extensions of 4D and Quadrays}
\author{Daniel Ari Friedman\\ ORCID: 0000-0001-6232-9096\\ Email: daniel@activeinference.institute}
\date{August 16, 2025}

\begin{document}
\maketitle

{
\hypersetup{linkcolor=black}
\setcounter{tocdepth}{3}
\tableofcontents
}
\setstretch{1.0}
\hypertarget{extensions-of-4d-and-quadrays}{%
\section{Extensions of 4D and
Quadrays}\label{extensions-of-4d-and-quadrays}}

Here we review some extensions of the Quadray 4D framework, including
multi-objective optimization, machine learning, computer graphics and
GPU acceleration, active inference, complex systems, pedagogy, and
implementations, with an emphasis on cognitive security.

\hypertarget{multi-objective-optimization}{%
\subsection{Multi-Objective
Optimization}\label{multi-objective-optimization}}

\begin{itemize}
\tightlist
\item
  Simplex faces encode trade-offs; integer volume measures solution
  diversity.
\item
  Pareto front exploration via tetrahedral traversal.
\end{itemize}

\hypertarget{machine-learning-and-robustness}{%
\subsection{Machine Learning and
Robustness}\label{machine-learning-and-robustness}}

\begin{itemize}
\tightlist
\item
  \textbf{Geometric regularization}: Quadray-constrained
  weights/topologies yield structural priors and improved stability.
\item
  \textbf{Adversarial robustness}: Discrete lattice projection reduces
  vulnerability to gradient-based adversarial perturbations by limiting
  directions.
\item
  \textbf{Ensembles}: Tetrahedral vertex voting and consensus improve
  robustness.
\end{itemize}

References: see
\href{https://en.wikipedia.org/wiki/Fisher_information}{Fisher
information},
\href{https://en.wikipedia.org/wiki/Natural_gradient}{Natural gradient},
and quadray conversion notes by Urner for embedding choices.

\hypertarget{computer-graphics-and-gpu-acceleration}{%
\subsection{Computer Graphics and GPU
Acceleration}\label{computer-graphics-and-gpu-acceleration}}

\begin{itemize}
\tightlist
\item
  \textbf{Quadray visualization acceleration}: GPU-accelerated rendering
  of tetrahedral coordinate systems enables real-time exploration of 4D
  geometric structures. The parallel nature of GPU architectures
  naturally maps to the four-basis vector representation of quadrays,
  allowing simultaneous computation of vertex positions, edge
  connections, and face tessellations across thousands of tetrahedra.
\item
  \textbf{Integer arithmetic optimization}: GPU compute shaders excel at
  integer-based volume calculations and determinant computations using
  the Bareiss algorithm. The discrete lattice structure of quadray
  coordinates benefits from parallel integer arithmetic units, achieving
  significant speedups over CPU implementations for large-scale
  geometric computations.
\item
  \textbf{Dynamic programming acceleration}: GPU-accelerated dynamic
  programming algorithms leverage CUDA Dynamic Parallelism for adaptive
  parallel computation of recursive geometric algorithms. This approach
  enables efficient handling of varying computational workloads in
  tetrahedral decomposition and optimization problems, as demonstrated
  in applications like the Mandelbrot set computation where dynamic
  parallelism manages computational complexity effectively.
\item
  \textbf{Parallel geometric algorithms}: Implementation of
  GPU-optimized versions of algorithms like QuickHull for convex hull
  computation in quadray space achieves substantial performance
  improvements. The tetrahedral lattice structure naturally supports
  parallel prefix sum operations and efficient neighbor queries,
  enabling real-time visualization of complex 4D geometric
  transformations.
\item
  \textbf{Memory bandwidth optimization}: The structured memory access
  patterns of quadray coordinates align well with GPU memory
  hierarchies, enabling efficient coalesced memory access for
  large-scale geometric datasets. This optimization is particularly
  beneficial for applications requiring real-time rendering of complex
  polyhedral structures and dynamic tessellations.
\end{itemize}

References: GPU-accelerated geometry processing techniques
(\href{https://arxiv.org/abs/1501.04706?utm_source=openai}{arxiv.org}),
CUDA Dynamic Parallelism for adaptive computation
(\href{https://developer.nvidia.com/blog/introduction-cuda-dynamic-parallelism/?utm_source=openai}{developer.nvidia.com}),
and parallel scan algorithms for optimization
(\href{https://developer.nvidia.com/gpugems/gpugems3/part-vi-gpu-computing?utm_source=openai}{developer.nvidia.com}).

\hypertarget{active-inference-and-free-energy}{%
\subsection{Active Inference and Free
Energy}\label{active-inference-and-free-energy}}

\begin{itemize}
\tightlist
\item
  Free energy
  \(\mathcal{F} = -\log P(o\mid s) + \mathrm{KL}[Q(s)\,\|\,P(s)]\) (see
  Eq. \eqref{eq:supp_free_energy} in the equations appendix);
  background:
  \href{https://en.wikipedia.org/wiki/Free_energy_principle}{Free energy
  principle} and overviews connecting to predictive coding and control.
\item
  Belief updates follow steepest descent in Fisher geometry using the
  natural gradient (see Eq. \eqref{eq:supp_natgrad} in the equations
  appendix); quadray constraints improve stability/interpretability.
\item
  Links to metabolic efficiency and biologically plausible computation.
\item
  For more information, see the Appendix: The Free Energy Principle and
  Active Inference.
\end{itemize}

\hypertarget{complex-systems-and-collective-intelligence}{%
\subsection{Complex Systems and Collective
Intelligence}\label{complex-systems-and-collective-intelligence}}

\begin{itemize}
\tightlist
\item
  Tetrahedral interaction patterns support distributed consensus and
  emergent behavior.
\item
  Resource allocation and network flows benefit from geometric
  constraints.
\item
  \textbf{Cognitive security}: Applying cognitive security can safeguard
  distributed consensus mechanisms from manipulation, preserving the
  reliability of emergent behaviors in complex systems. Incorporating
  cognitive security measures can protect the integrity of belief
  updates and decision-making processes, ensuring that actions are based
  on accurate and unmanipulated information.
\end{itemize}

\hypertarget{geospatial-intelligence-and-the-world-game}{%
\subsection{Geospatial Intelligence and the World
Game}\label{geospatial-intelligence-and-the-world-game}}

\begin{itemize}
\tightlist
\item
  \textbf{Spatial data integration}: Quadray tetrahedral frameworks
  provide natural tessellations for geospatial data analysis, where the
  Dymaxion projection's minimal distortion aligns with Fuller's World
  Game objectives of holistic global perspective. The tetrahedral
  lattice supports efficient spatial indexing and neighbor queries for
  distributed geospatial intelligence operations.
\item
  \textbf{Resource allocation optimization}: The World Game's goal of
  ``making the world work for 100\% of humanity'' translates to
  multi-objective optimization problems where tetrahedral simplex faces
  encode trade-offs between population centers, resource distribution,
  and ecological constraints. Integer volume quantization ensures
  discrete, interpretable solutions for global resource allocation.
\item
  \textbf{Cognitive security in distributed sensing}: Geospatial
  intelligence networks benefit from tetrahedral consensus mechanisms
  that resist manipulation of spatial data streams. The geometric
  constraints of Fuller.4D provide natural validation frameworks for
  detecting anomalous spatial patterns and maintaining data integrity
  across distributed sensor networks.
\item
  \textbf{Tetrahedral tessellations for global modeling}: The World
  Game's emphasis on interconnected global systems maps naturally to
  tetrahedral decompositions of the Dymaxion projection, where each
  tetrahedron represents a coherent region for local optimization while
  maintaining global connectivity through shared faces and edges.
\end{itemize}

\hypertarget{quadrays-synergetics-fuller.4d-and-william-blake}{%
\subsection{Quadrays, Synergetics (Fuller.4D), and William
Blake}\label{quadrays-synergetics-fuller.4d-and-william-blake}}

\begin{itemize}
\tightlist
\item
  Quadrays (tetrahedral coordinates) instantiate Fuller's Synergetics
  emphasis on the tetrahedron as a structural primitive; in this
  manuscript's terminology this corresponds to Fuller.4D. Tetrahedral
  frames support part--whole reasoning and efficient decompositions used
  throughout.
\item
  William Blake's ``fourfold vision'' (single, twofold, threefold,
  fourfold) provides a historical metaphor for multiscale perception and
  inference. Read through Fisher geometry and natural gradient dynamics,
  it parallels multilayer predictive processing and counterfactual
  simulation. For background, see a concise overview of Blake's
  visionary psycho‑topographies in British Art Studies
  (\href{https://www.britishartstudies.ac.uk/index/article-index/visionary-sense-of-london/article-category/cover-collaboration}{visionary
  art analysis}) and the Active Inference Institute's MathArt Stream \#8
  (\href{https://zenodo.org/records/13711302}{Active Inference \&
  Blake}).
\item
  Juxtaposing Blake and Fuller foregrounds ``comprehensivity'': holistic
  design and sensemaking via geometric primitives. Context:
  (\href{https://zenodo.org/records/7519132}{Fuller \& Blake: Lives in
  Juxtaposition}) and pedagogical antecedents in experimental design
  education at Black Mountain College
  (\href{https://commons.princeton.edu/eng574-s23/wp-content/uploads/sites/348/2023/03/Diaz-The-Experimenters-Chance-and-Design-at-Black-Mountain-College.pdf}{Diaz,
  Chance and Design at Black Mountain College -- PDF}).
\item
  Implications for Quadray practice: four‑facet summaries of
  models/trajectories, tetrahedral consensus in ensembles, and
  stigmergic annotation patterns for cognitive security and distributed
  sensemaking.
\end{itemize}

\hypertarget{pedagogy-and-implementations}{%
\subsection{Pedagogy and
Implementations}\label{pedagogy-and-implementations}}

Kirby Urner's comprehensive
\href{https://github.com/4dsolutions}{4dsolutions ecosystem} provides
extensive educational resources and cross-platform implementations for
Quadray computation and visualization. For comprehensive details on
educational frameworks, cross-language implementations, historical
context, and community development, see the
\href{07_resources.md}{Resources} section.

\hypertarget{higher-dimensions-and-decompositions}{%
\subsection{Higher Dimensions and
Decompositions}\label{higher-dimensions-and-decompositions}}

\begin{itemize}
\tightlist
\item
  Decompose higher-dimensional simplexes into tetrahedra; sum integer
  volumes to maintain quantization.
\item
  Tessellations support parallel/distributed implementations.
\end{itemize}

\hypertarget{limitations-and-future-work}{%
\subsection{Limitations and Future
Work}\label{limitations-and-future-work}}

\begin{itemize}
\tightlist
\item
  Benchmark breadth: extend beyond convex/quadratic toys to real tasks
  (registration, robust regression, control) with ablations.
\item
  Distance sensitivity: compare embeddings and their effect on optimizer
  trajectories; document recommended defaults.
\item
  Hybrid schemes: study schedules that interleave continuous proposals
  with lattice projection.
\end{itemize}

\end{document}
