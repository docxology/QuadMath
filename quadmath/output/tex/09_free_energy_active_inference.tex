% Options for packages loaded elsewhere
\PassOptionsToPackage{unicode}{hyperref}
\PassOptionsToPackage{hyphens}{url}
\PassOptionsToPackage{dvipsnames,svgnames*,x11names*}{xcolor}
%
\documentclass[
  10pt,
]{article}
\usepackage{lmodern}
\usepackage{setspace}
\usepackage{amssymb,amsmath}
\usepackage{ifxetex,ifluatex}
\ifnum 0\ifxetex 1\fi\ifluatex 1\fi=0 % if pdftex
  \usepackage[T1]{fontenc}
  \usepackage[utf8]{inputenc}
  \usepackage{textcomp} % provide euro and other symbols
\else % if luatex or xetex
  \usepackage{unicode-math}
  \defaultfontfeatures{Scale=MatchLowercase}
  \defaultfontfeatures[\rmfamily]{Ligatures=TeX,Scale=1}
  \setmainfont[]{DejaVu Serif}
  \setmonofont[]{DejaVu Sans Mono}
\fi
% Use upquote if available, for straight quotes in verbatim environments
\IfFileExists{upquote.sty}{\usepackage{upquote}}{}
\IfFileExists{microtype.sty}{% use microtype if available
  \usepackage[]{microtype}
  \UseMicrotypeSet[protrusion]{basicmath} % disable protrusion for tt fonts
}{}
\makeatletter
\@ifundefined{KOMAClassName}{% if non-KOMA class
  \IfFileExists{parskip.sty}{%
    \usepackage{parskip}
  }{% else
    \setlength{\parindent}{0pt}
    \setlength{\parskip}{6pt plus 2pt minus 1pt}}
}{% if KOMA class
  \KOMAoptions{parskip=half}}
\makeatother
\usepackage{xcolor}
\IfFileExists{xurl.sty}{\usepackage{xurl}}{} % add URL line breaks if available
\IfFileExists{bookmark.sty}{\usepackage{bookmark}}{\usepackage{hyperref}}
\hypersetup{
  colorlinks=true,
  linkcolor=red,
  filecolor=red,
  citecolor=red,
  urlcolor=red,
  pdfcreator={LaTeX via pandoc}}
\urlstyle{same} % disable monospaced font for URLs
\usepackage[margin=1cm,top=1cm,bottom=1cm,left=1cm,right=1cm,includeheadfoot]{geometry}
\usepackage{listings}
\newcommand{\passthrough}[1]{#1}
\lstset{defaultdialect=[5.3]Lua}
\lstset{defaultdialect=[x86masm]Assembler}
\usepackage{graphicx}
\makeatletter
\def\maxwidth{\ifdim\Gin@nat@width>\linewidth\linewidth\else\Gin@nat@width\fi}
\def\maxheight{\ifdim\Gin@nat@height>\textheight\textheight\else\Gin@nat@height\fi}
\makeatother
% Scale images if necessary, so that they will not overflow the page
% margins by default, and it is still possible to overwrite the defaults
% using explicit options in \includegraphics[width, height, ...]{}
\setkeys{Gin}{width=\maxwidth,height=\maxheight,keepaspectratio}
% Set default figure placement to htbp
\makeatletter
\def\fps@figure{htbp}
\makeatother
\setlength{\emergencystretch}{3em} % prevent overfull lines
\providecommand{\tightlist}{%
  \setlength{\itemsep}{0pt}\setlength{\parskip}{0pt}}
\setcounter{secnumdepth}{3}
% Enable graphics inclusion and ensure figure numbering works
\usepackage{graphicx}
\renewcommand{\figurename}{Figure}

% Configure fonts for Unicode support with fallbacks
\usepackage{newunicodechar}
\newunicodechar{⁴}{\textsuperscript{4}}
\newunicodechar{₄}{\textsubscript{4}}

% Enhanced code block styling for better contrast and readability
\usepackage{fancyvrb}
\usepackage{xcolor}
\usepackage{listings}

% Define custom colors for code blocks
\definecolor{codebg}{RGB}{245, 245, 245}      % Light gray background
\definecolor{codeborder}{RGB}{200, 200, 200}  % Medium gray border
\definecolor{codefg}{RGB}{50, 50, 50}         % Dark gray text

% Configure Verbatim environment for inline code
\DefineVerbatimEnvironment{Verbatim}{Verbatim}{%
    fontsize=\small,
    frame=single,
    framerule=0.5pt,
    framesep=3pt,
    rulecolor=\color{codeborder},
    bgcolor=\color{codebg},
    fgcolor=\color{codefg}
}

% Configure code block styling
\DefineVerbatimEnvironment{Highlighting}{Verbatim}{%
    fontsize=\footnotesize,
    frame=single,
    framerule=0.5pt,
    framesep=5pt,
    rulecolor=\color{codeborder},
    bgcolor=\color{codebg},
    fgcolor=\color{codefg}
}

% Style inline code with \texttt
\renewcommand{\texttt}[1]{%
    \colorbox{codebg}{\color{codefg}\ttfamily #1}%
}

% Configure listings package for code blocks
\lstset{
    backgroundcolor=\color{codebg},
    basicstyle=\footnotesize\ttfamily\color{codefg},
    breakatwhitespace=false,
    breaklines=true,
    captionpos=b,
    commentstyle=\color{codefg},
    deletekeywords={...},
    escapeinside={\%*}{*)},
    extendedchars=true,
    frame=single,
    framerule=0.5pt,
    framesep=5pt,
    keepspaces=true,
    keywordstyle=\color{codefg},
    language=Python,
    morekeywords={*,...},
    numbers=left,
    numbersep=5pt,
    numberstyle=\tiny\color{codefg},
    rulecolor=\color{codeborder},
    showspaces=false,
    showstringspaces=false,
    showtabs=false,
    stepnumber=1,
    stringstyle=\color{codefg},
    tabsize=2,
    title=\lstname
}

% Override any Pandoc default lstset configurations
\AtBeginDocument{
    \lstset{
        backgroundcolor=\color{codebg},
        basicstyle=\footnotesize\ttfamily\color{codefg},
        frame=single,
        framerule=0.5pt,
        framesep=5pt,
        rulecolor=\color{codeborder},
        numbers=left,
        numbersep=5pt,
        numberstyle=\tiny\color{codefg}
    }
}

% Configure hyperref colors consistently
\AtBeginDocument{
% Override pandoc's hidelinks setting with consistent options
\hypersetup{
    colorlinks=true,
    allcolors=red,
    linkcolor=red,
    urlcolor=red,
    citecolor=red,
    filecolor=red,
    menucolor=red,
    linktoc=all
}
}

% Simple page break support for document structure

\title{Appendix: Free Energy and Active Inference}
\author{Daniel Ari Friedman\\ ORCID: 0000-0001-6232-9096\\ Email: daniel@activeinference.institute}
\date{August 16, 2025}

\begin{document}
\maketitle

{
\hypersetup{linkcolor=black}
\setcounter{tocdepth}{3}
\tableofcontents
}
\setstretch{1.0}
\hypertarget{appendix-the-free-energy-principle-and-active-inference}{%
\section{Appendix: The Free Energy Principle and Active
Inference}\label{appendix-the-free-energy-principle-and-active-inference}}

\hypertarget{overview}{%
\subsection{Overview}\label{overview}}

The Free Energy Principle (FEP) posits that biological systems maintain
their states by minimizing variational free energy, thereby reducing
surprise via prediction and model updating. Active Inference extends
this by casting action selection as inference under prior preferences.
Background: see the concise overview on the
\href{https://en.wikipedia.org/wiki/Free_energy_principle}{Free energy
principle} and the monograph
\href{https://direct.mit.edu/books/oa-monograph/5299/Active-InferenceThe-Free-Energy-Principle-in-Mind}{Active
Inference (MIT Press)}.

This appendix emphasizes relationships among: (i) the four-fold
partition of Active Inference, (ii) Quadrays (Fuller.4D) as a geometric
scaffold for mapping this partition, and (iii) information-geometric
flows (Einstein.4D analogy) that underpin perception--action updates.
For the naming of 4D namespaces used throughout---Coxeter.4D (Euclidean
E4), Einstein.4D (Minkowski spacetime analogy), Fuller.4D
(Synergetics/Quadrays)---see
\passthrough{\lstinline!02\_4d\_namespaces.md!}.

\hypertarget{mathematical-formulation-and-equation-callouts-equations-linkage}{%
\subsection{Mathematical Formulation and Equation Callouts (Equations
linkage)}\label{mathematical-formulation-and-equation-callouts-equations-linkage}}

\begin{itemize}
\item
  Variational free energy (discrete states) --- see Eq.
  \eqref{eq:free_energy} in the equations appendix, implemented by
  \href{08_equations_appendix.md\#code:free_energy}{\passthrough{\lstinline!free\_energy!}}.
\item
  Fisher Information Matrix (FIM) as metric --- see Eq. \eqref{eq:fim}
  in the equations appendix and
  \href{08_equations_appendix.md\#code:fisher_information_matrix}{\passthrough{\lstinline!fisher\_information\_matrix!}}.
\item
  Natural gradient descent under information geometry --- see Eq.
  \eqref{eq:natural_gradient} in the equations appendix and
  \href{08_equations_appendix.md\#code:natural_gradient_step}{\passthrough{\lstinline!natural\_gradient\_step!}};
  overview:
  \href{https://en.wikipedia.org/wiki/Natural_gradient}{Natural
  gradient}.
\end{itemize}

Figures: The following Active Inference figures demonstrate the
integration of natural gradient descent with Active Inference principles
and the 4D framework context.

Discrete variational optimization on the quadray lattice:
\passthrough{\lstinline!discrete\_ivm\_descent!} greedily descends a
free-energy-like objective over IVM moves, yielding integer-valued
trajectories. See the path animation artifact
\passthrough{\lstinline!discrete\_path.mp4!} in
\passthrough{\lstinline!quadmath/output/!}.

\begin{figure}
\centering
\includegraphics{../output/figures/partition_tetrahedron.png}
\caption{\textbf{Active Inference four-fold partition mapped to a
Quadray tetrahedron in Fuller.4D}. This 3D tetrahedral visualization
demonstrates the geometric embedding of Active Inference's fundamental
four-fold partition within the Quadray coordinate system.
\textbf{Tetrahedral structure}: The four vertices of the regular
tetrahedron represent the four components of the Active Inference
framework: perception, action, internal states, and external states.
\textbf{Partition mapping}: Each face of the tetrahedron corresponds to
a specific partition of the four-fold system, with the edges
representing the relationships and interactions between different
components. \textbf{Fuller.4D significance}: This geometric
representation leverages the tetrahedral nature of Quadray coordinates
to provide an intuitive visualization of the Active Inference
framework's structure. The tetrahedron serves as a natural container for
the four-fold partition, emphasizing the interconnected nature of
perception, action, and state representation in active inference.
\textbf{Optimization context}: The tetrahedral geometry also suggests
natural optimization strategies that respect the four-fold structure,
potentially leading to more efficient inference algorithms that leverage
the geometric relationships between different components. This
visualization demonstrates how the Fuller.4D framework can provide
insights into complex systems like Active Inference through geometric
intuition.}
\end{figure}

\begin{figure}
\centering
\includegraphics{../output/figures/figure_13_4d_trajectory.png}
\caption{\textbf{4D Natural Gradient Trajectory with Active Inference
Context}. This comprehensive visualization demonstrates natural gradient
descent operating within the Active Inference framework, showing how
information-geometric optimization drives perception-action dynamics.
\textbf{3D Trajectory}: The main panel shows the 4D parameter evolution
in 3D space with time encoded as color, representing the four-fold
partition of Active Inference: perception (μ), action (a), internal
states (s), and external causes (ψ). \textbf{Free Energy Evolution}: The
right panel tracks free energy minimization over optimization steps,
demonstrating the Active Inference principle of surprise reduction.
\textbf{Component Dynamics}: The bottom-left panel shows how each
component of the four-fold partition evolves during optimization,
revealing the coordinated dynamics of perception and action.
\textbf{Optimization Diagnostics}: The bottom-center panel displays step
sizes and gradient norms, providing insights into the convergence
behavior and numerical stability of the natural gradient algorithm.
\textbf{Fisher Information}: The bottom-right panel displays the Fisher
Information Matrix that guides natural gradient descent, showing the
information geometry underlying the optimization process. This figure
demonstrates how natural gradient descent implements geodesic motion on
the information manifold, analogous to how particles follow geodesics in
Einstein.4D spacetime, while operating within the tetrahedral structure
of Fuller.4D coordinates. The optimization now shows stable convergence
in just 11 steps with final parameter errors below 0.015, demonstrating
the effectiveness of information-geometric optimization in Active
Inference frameworks.}
\end{figure}

\begin{figure}
\centering
\includegraphics{../output/figures/figure_14_free_energy_landscape.png}
\caption{\textbf{Free Energy Landscape in 4D Active Inference
Framework}. This comprehensive visualization explores the variational
free energy surface over perception and action parameters. \textbf{3D
landscape}: The surface plot shows the free energy as a function of two
variational parameters, revealing the complex topology that Active
Inference optimization must navigate. \textbf{Contour analysis}: 2D
contours provide detailed information about parameter sensitivity and
optimization paths. \textbf{Cross-sectional analysis}: Multiple
cross-sections at different parameter values demonstrate how free energy
varies with respect to individual parameters, revealing the landscape's
structure. \textbf{Four-fold partition visualization}: The text panel
explains how Active Inference maps to tetrahedral structures in
Fuller.4D, with the four components (μ, s, a, ψ) representing internal
states, sensory observations, actions, and external causes.
\textbf{Information geometry metrics}: Local curvature analysis reveals
the Fisher information structure, showing how the information manifold's
geometry influences optimization dynamics. \textbf{Mathematical
foundation}: The visualization demonstrates the mathematical structure
of variational inference, including variational posteriors Q(s), priors
P(s), and likelihoods P(o\textbar s) that connect observations to latent
states.}
\end{figure}

\hypertarget{four-fold-partition-and-tetrahedral-mapping-quadrays-fuller.4d}{%
\subsection{Four-Fold Partition and Tetrahedral Mapping (Quadrays;
Fuller.4D)}\label{four-fold-partition-and-tetrahedral-mapping-quadrays-fuller.4d}}

Active Inference partitions the agent--environment system into four
coupled states:

\begin{itemize}
\tightlist
\item
  Internal (\(\mu\)) --- agent's internal states
\item
  Sensory (\(s\)) --- observations
\item
  Active (\(a\)) --- actions
\item
  External (\(\psi\)) --- latent environmental causes
\end{itemize}

See, for an overview of this partition and generative process
formulations, the
\href{https://discovery.ucl.ac.uk/id/eprint/10176959/1/1-s2.0-S1571064523001094-main.pdf}{Active
Inference review} and the general entry on
\href{https://en.wikipedia.org/wiki/Active_inference}{Active inference}.

Tetrahedral mapping via Quadrays (Fuller.4D): assign each state to a
vertex of a tetrahedron, using Quadray coordinates
\passthrough{\lstinline!(A,B,C,D)!} with non-negative components and at
least one zero after normalization. One canonical mapping is
\passthrough{\lstinline!A \\leftrightarrow Internal (\\mu)!},
\passthrough{\lstinline!B \\leftrightarrow Sensory (s)!},
\passthrough{\lstinline!C \\leftrightarrow Active (a)!},
\passthrough{\lstinline!D \\leftrightarrow External (\\psi)!}. The edges
capture the pairwise couplings (e.g.,
\passthrough{\lstinline!\\mu\\text\{--\}s!} for perceptual inference;
\passthrough{\lstinline!a\\text\{--\}\\psi!} for control). Integer
tetravolume then quantifies the ``coupled capacity'' region spanned by
jointly feasible states in a time slice; see
\passthrough{\lstinline!Quadray!} and tetravolume methods in
\passthrough{\lstinline!03\_quadray\_methods.md!}.

Interpretation note: this Quadray-based mapping is a didactic geometric
scaffold. It is not standard in the Active Inference literature, which
typically develops the four-state partition in probabilistic graphical
terms. Our use highlights structural symmetries and discrete volumetric
quantities available in Fuller.4D, building on the computational
foundations developed in the
\href{https://github.com/4dsolutions}{4dsolutions ecosystem} for
tetrahedral modeling and volume calculations. See the
\href{07_resources.md}{Resources} section for comprehensive details on
the computational implementations.

Code linkage (no snippet): see
\passthrough{\lstinline!example\_partition\_tetra\_volume!} in
\passthrough{\lstinline!src/examples.py!} and the partition tetrahedron
figure above.

\hypertarget{how-the-4d-namespaces-relate-here}{%
\subsection{How the 4D namespaces relate
here}\label{how-the-4d-namespaces-relate-here}}

\begin{itemize}
\tightlist
\item
  Fuller.4D (Quadrays): geometric embedding of the four-state partition
  on a tetrahedron; integer tetravolumes and IVM moves provide discrete
  combinatorial structure.
\item
  Coxeter.4D (Euclidean E4): exact Euclidean measurements (e.g.,
  Cayley--Menger determinants) for tetrahedra underlying volumetric
  comparisons and scale relations.
\item
  Einstein.4D (Minkowski analogy): information-geometric flows (natural
  gradient, metric-aware updates) supply a continuum picture for
  perception--action dynamics.
\end{itemize}

The three roles are complementary: Fuller.4D encodes partition
structure, Coxeter.4D provides exact metric geometry for static
comparisons, and Einstein.4D guides dynamical descent.

\hypertarget{joint-optimization-in-the-tetrahedral-framework-methods-linkage}{%
\subsection{Joint Optimization in the Tetrahedral Framework (Methods
linkage)}\label{joint-optimization-in-the-tetrahedral-framework-methods-linkage}}

\begin{itemize}
\tightlist
\item
  Perception: update \(\mu\) to minimize prediction error on \(s\) under
  the generative model (descending \(\nabla_{\mu} F\)).
\item
  Action: select \(a\) that steers \(\psi\) toward preferred outcomes
  (descending \(\nabla_{a} F\)).
\end{itemize}

Continuous-time flows (Einstein.4D analogy for metric/geodesic
intuition): see \passthrough{\lstinline!perception\_update!} and
\passthrough{\lstinline!action\_update!} in
\passthrough{\lstinline!src/information.py!}. Discrete Quadray moves
connect to these flows via greedy descent on a local free-energy-like
objective; see \passthrough{\lstinline!discrete\_ivm\_descent!} in
\passthrough{\lstinline!src/discrete\_variational.py!} and the path
artifacts in \passthrough{\lstinline!quadmath/output/!}.

\hypertarget{implications-for-ai-and-robust-computation}{%
\subsection{Implications for AI and Robust
Computation}\label{implications-for-ai-and-robust-computation}}

FEP/Active Inference provide algorithms that unify perception and action
under uncertainty, offering biologically plausible alternatives to
standard RL with adaptive exploration and robust decision-making. See
\href{https://arxiv.org/abs/1907.03876}{applications in AI
(arXiv:1907.03876)}.

\hypertarget{code-reproducibility-and-cross-references}{%
\subsection{Code, Reproducibility, and
Cross-References}\label{code-reproducibility-and-cross-references}}

-- Equation references:
\href{08_equations_appendix.md\#eq:free_energy}{Eq. (Free Energy)},
\href{08_equations_appendix.md\#eq:fim}{Eq. (FIM)},
\href{08_equations_appendix.md\#eq:natgrad}{Eq. (Natural Gradient)} in
\passthrough{\lstinline!08\_equations\_appendix.md!}. -- Code anchors
(for readers who want to run experiments):
\href{03_quadray_methods.md\#code:free_energy}{\passthrough{\lstinline!free\_energy!}},
\href{03_quadray_methods.md\#code:fisher_information_matrix}{\passthrough{\lstinline!fisher\_information\_matrix!}},
\href{03_quadray_methods.md\#code:natural_gradient_step}{\passthrough{\lstinline!natural\_gradient\_step!}},
\passthrough{\lstinline!perception\_update!},
\passthrough{\lstinline!action\_update!}, and
\passthrough{\lstinline!discrete\_ivm\_descent!} in
\passthrough{\lstinline!src/information.py!} and
\passthrough{\lstinline!src/discrete\_variational.py!}.

Demo and figures generated by
\passthrough{\lstinline!quadmath/scripts/information\_demo.py!} and
\passthrough{\lstinline!quadmath/scripts/active\_inference\_figures.py!}
output to \passthrough{\lstinline!quadmath/output/!}:

\begin{itemize}
\tightlist
\item
  \textbf{Active Inference Visualizations}:
  \passthrough{\lstinline!figure\_13\_4d\_trajectory.png!},
  \passthrough{\lstinline!figure\_14\_free\_energy\_landscape.png!}
  demonstrating 4D framework integration
\item
  \textbf{Information Geometry Visualizations}:
  \passthrough{\lstinline!fisher\_information\_matrix.png!},
  \passthrough{\lstinline!fisher\_information\_eigenspectrum.png!},
  \passthrough{\lstinline!natural\_gradient\_path.png!},
  \passthrough{\lstinline!free\_energy\_curve.png!},
  \passthrough{\lstinline!partition\_tetrahedron.png!}
\item
  \textbf{Raw data}: \passthrough{\lstinline!figure\_13\_data.npz!},
  \passthrough{\lstinline!figure\_14\_data.npz!},
  \passthrough{\lstinline!fisher\_information\_matrix.csv!},
  \passthrough{\lstinline!fisher\_information\_matrix.npz!} (F, grads,
  X, y, w\_true, w\_est),
  \passthrough{\lstinline!fisher\_information\_eigenvalues.csv!},
  \passthrough{\lstinline!fisher\_information\_eigensystem.npz!}
\item
  \textbf{External validation}: Cross-reference with volume calculations
  and tetrahedral modeling tools from the
  \href{https://github.com/4dsolutions}{4dsolutions ecosystem}. See the
  \href{07_resources.md}{Resources} section for comprehensive details.
\end{itemize}

\end{document}
