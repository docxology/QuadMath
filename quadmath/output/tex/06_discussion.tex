% Options for packages loaded elsewhere
\PassOptionsToPackage{unicode}{hyperref}
\PassOptionsToPackage{hyphens}{url}
\PassOptionsToPackage{dvipsnames,svgnames*,x11names*}{xcolor}
%
\documentclass[
  10pt,
]{article}
\usepackage{lmodern}
\usepackage{setspace}
\usepackage{amssymb,amsmath}
\usepackage{ifxetex,ifluatex}
\ifnum 0\ifxetex 1\fi\ifluatex 1\fi=0 % if pdftex
  \usepackage[T1]{fontenc}
  \usepackage[utf8]{inputenc}
  \usepackage{textcomp} % provide euro and other symbols
\else % if luatex or xetex
  \usepackage{unicode-math}
  \defaultfontfeatures{Scale=MatchLowercase}
  \defaultfontfeatures[\rmfamily]{Ligatures=TeX,Scale=1}
  \setmainfont[]{DejaVu Serif}
  \setmonofont[]{DejaVu Sans Mono}
\fi
% Use upquote if available, for straight quotes in verbatim environments
\IfFileExists{upquote.sty}{\usepackage{upquote}}{}
\IfFileExists{microtype.sty}{% use microtype if available
  \usepackage[]{microtype}
  \UseMicrotypeSet[protrusion]{basicmath} % disable protrusion for tt fonts
}{}
\makeatletter
\@ifundefined{KOMAClassName}{% if non-KOMA class
  \IfFileExists{parskip.sty}{%
    \usepackage{parskip}
  }{% else
    \setlength{\parindent}{0pt}
    \setlength{\parskip}{6pt plus 2pt minus 1pt}}
}{% if KOMA class
  \KOMAoptions{parskip=half}}
\makeatother
\usepackage{xcolor}
\IfFileExists{xurl.sty}{\usepackage{xurl}}{} % add URL line breaks if available
\IfFileExists{bookmark.sty}{\usepackage{bookmark}}{\usepackage{hyperref}}
\hypersetup{
  colorlinks=true,
  linkcolor=red,
  filecolor=red,
  citecolor=red,
  urlcolor=red,
  pdfcreator={LaTeX via pandoc}}
\urlstyle{same} % disable monospaced font for URLs
\usepackage[margin=1cm,top=1cm,bottom=1cm,left=1cm,right=1cm,includeheadfoot]{geometry}
\usepackage{listings}
\newcommand{\passthrough}[1]{#1}
\lstset{defaultdialect=[5.3]Lua}
\lstset{defaultdialect=[x86masm]Assembler}
\setlength{\emergencystretch}{3em} % prevent overfull lines
\providecommand{\tightlist}{%
  \setlength{\itemsep}{0pt}\setlength{\parskip}{0pt}}
\setcounter{secnumdepth}{3}
% Enable graphics inclusion and ensure figure numbering works
\usepackage{graphicx}
\renewcommand{\figurename}{Figure}

% Configure fonts for Unicode support with fallbacks
\usepackage{newunicodechar}
\newunicodechar{⁴}{\textsuperscript{4}}
\newunicodechar{₄}{\textsubscript{4}}

% Enhanced code block styling for better contrast and readability
\usepackage{fancyvrb}
\usepackage{xcolor}
\usepackage{listings}

% Define custom colors for code blocks
\definecolor{codebg}{RGB}{245, 245, 245}      % Light gray background
\definecolor{codeborder}{RGB}{200, 200, 200}  % Medium gray border
\definecolor{codefg}{RGB}{50, 50, 50}         % Dark gray text

% Configure Verbatim environment for inline code
\DefineVerbatimEnvironment{Verbatim}{Verbatim}{%
    fontsize=\small,
    frame=single,
    framerule=0.5pt,
    framesep=3pt,
    rulecolor=\color{codeborder},
    bgcolor=\color{codebg},
    fgcolor=\color{codefg}
}

% Configure code block styling
\DefineVerbatimEnvironment{Highlighting}{Verbatim}{%
    fontsize=\footnotesize,
    frame=single,
    framerule=0.5pt,
    framesep=5pt,
    rulecolor=\color{codeborder},
    bgcolor=\color{codebg},
    fgcolor=\color{codefg}
}

% Style inline code with \texttt
\renewcommand{\texttt}[1]{%
    \colorbox{codebg}{\color{codefg}\ttfamily #1}%
}

% Configure listings package for code blocks
\lstset{
    backgroundcolor=\color{codebg},
    basicstyle=\footnotesize\ttfamily\color{codefg},
    breakatwhitespace=false,
    breaklines=true,
    captionpos=b,
    commentstyle=\color{codefg},
    deletekeywords={...},
    escapeinside={\%*}{*)},
    extendedchars=true,
    frame=single,
    framerule=0.5pt,
    framesep=5pt,
    keepspaces=true,
    keywordstyle=\color{codefg},
    language=Python,
    morekeywords={*,...},
    numbers=left,
    numbersep=5pt,
    numberstyle=\tiny\color{codefg},
    rulecolor=\color{codeborder},
    showspaces=false,
    showstringspaces=false,
    showtabs=false,
    stepnumber=1,
    stringstyle=\color{codefg},
    tabsize=2,
    title=\lstname
}

% Override any Pandoc default lstset configurations
\AtBeginDocument{
    \lstset{
        backgroundcolor=\color{codebg},
        basicstyle=\footnotesize\ttfamily\color{codefg},
        frame=single,
        framerule=0.5pt,
        framesep=5pt,
        rulecolor=\color{codeborder},
        numbers=left,
        numbersep=5pt,
        numberstyle=\tiny\color{codefg}
    }
}

% Configure hyperref colors consistently
\AtBeginDocument{
% Override pandoc's hidelinks setting with consistent options
\hypersetup{
    colorlinks=true,
    allcolors=red,
    linkcolor=red,
    urlcolor=red,
    citecolor=red,
    filecolor=red,
    menucolor=red,
    linktoc=all
}
}

% Simple page break support for document structure
% Note: Page breaks are handled in the markdown generation, not here

\title{Discussion}
\author{Daniel Ari Friedman\\ ORCID: 0000-0001-6232-9096\\ Email: daniel@activeinference.institute}
\date{August 15, 2025}

\begin{document}
\maketitle

{
\hypersetup{linkcolor=black}
\setcounter{tocdepth}{3}
\tableofcontents
}
\setstretch{1.0}
\hypertarget{discussion}{%
\section{Discussion}\label{discussion}}

Quadray geometry (Fuller.4D) offers an interpretable, quantized view of
geometry, topology, information, and optimization. Integer volumes
enforce discrete dynamics, acting as a structural prior that can
regularize optimization, reduce overfitting, prevent numerical
fragility, and enable integer-based accelerated methods. Information
geometry provides a right language for optimization in the synergetic
tradition: optimization proceeds not through arbitrary parameter-space
moves in continuous space, but along geodesics defined by information
content (see Eq. \eqref{eq:supp_fim} and Eq. \eqref{eq:supp_natgrad} in
the equations appendix; overview:
\href{https://en.wikipedia.org/wiki/Natural_gradient}{Natural
gradient}).

Limitations and considerations:

\begin{itemize}
\tightlist
\item
  \textbf{Embeddings and distances}: Mapping between quadray and
  Euclidean coordinates must be selected carefully for distance
  calculations.
\item
  \textbf{Hybrid strategies}: Some problems may require hybrid
  strategies (continuous steps with periodic lattice projection).
\item
  \textbf{Benchmarking}: Empirical benchmarking remains important to
  quantify benefits across domains.
\end{itemize}

In practical analysis and simulation, numerical precision matters.
Integer-volume reasoning is exact in theory, but empirical evaluation
(e.g., determinants, Fisher Information, geodesics) can benefit from
high-precision arithmetic. When double precision is insufficient,
quad-precision arithmetic (binary128) via GCC's
\passthrough{\lstinline!libquadmath!} provides the
\passthrough{\lstinline!\_\_float128!} type and a rich math API for
robust computation. See the official documentation for details on
functions and I/O:
\href{https://gcc.gnu.org/onlinedocs/libquadmath/index.html}{GCC
libquadmath}.

\hypertarget{fisher-information-and-curvature}{%
\subsection{Fisher Information and
Curvature}\label{fisher-information-and-curvature}}

The Fisher Information Matrix (FIM) defines a Riemannian metric on
parameter space and quantifies local curvature of the statistical
manifold. High curvature directions (large eigenvalues of
\passthrough{\lstinline!F!}) indicate parameters to which the model is
most sensitive; small eigenvalues indicate sloppy directions. Our
eigenspectrum visualization (see the Fisher Information Matrix
eigenspectrum figure above) highlights these scales. Background:
\href{https://en.wikipedia.org/wiki/Fisher_information}{Fisher
information}.

Implication: curvature-aware steps using Eq. \eqref{eq:supp_natgrad} in
the equations appendix adaptively scale updates by the inverse metric,
improving conditioning relative to vanilla gradient descent.

A curious connection unites geodesics in information geometry, the
physical principle of least action, and Buckminster Fuller's tensegrity
geodesic domes (Fuller.4D). On statistical manifolds, geodesics are
shortest paths under the Fisher metric, and natural-gradient flows
approximate least-action trajectories by minimizing an
information-length functional constrained by curvature (Eqs.
\eqref{eq:supp_fim}, \eqref{eq:supp_natgrad} in the equations appendix).
In tensegrity domes, geodesic lines on triangulated spherical shells
distribute stress nearly uniformly while the network balances continuous
tension with discontinuous compression, attaining maximal stiffness with
minimal material. Both systems exemplify constraint-balanced minimalism:
an extremal path emerges by trading off cost (action or information
length) against structure (metric curvature or tensegrity
compatibility). The shared economy---optimal routing through low-cost
directions---links geodesic shells in architecture to geodesic flows in
parameter spaces; see background on tensegrity/geodesic domes @Web.

\hypertarget{quadray-coordinates-and-4d-structure-fuller.4d-vs-coxeter.4d-vs-einstein.4d}{%
\subsection{Quadray Coordinates and 4D Structure (Fuller.4D vs
Coxeter.4D vs
Einstein.4D)}\label{quadray-coordinates-and-4d-structure-fuller.4d-vs-coxeter.4d-vs-einstein.4d}}

Quadray coordinates provide a tetrahedral basis with projective
normalization, aligning with close-packed sphere centers (IVM).
Symmetries common in quadray parameterizations often yield near
block-diagonal structure in \passthrough{\lstinline!F!}, simplifying
inversion and preconditioning. Overview:
\href{https://en.wikipedia.org/wiki/Quadray_coordinates}{Quadray
coordinates} and synergetics background. We stress the namespace
boundaries: (i) Fuller.4D for lattice and integer volumes, (ii)
Coxeter.4D for Euclidean embeddings, lengths, and simplex families,
(iii) Einstein.4D for metric analogies only --- not for interpreting
synergetic tetravolumes.

\hypertarget{integrating-fim-with-quadray-models}{%
\subsection{Integrating FIM with Quadray
Models}\label{integrating-fim-with-quadray-models}}

Applying the FIM within quadray-parameterized models ties statistical
curvature to tetrahedral structure. Practical takeaways:

\begin{itemize}
\tightlist
\item
  Use \passthrough{\lstinline!fisher\_information\_matrix!} to estimate
  \passthrough{\lstinline!F!} from per-sample gradients; inspect
  principal directions via \passthrough{\lstinline!fim\_eigenspectrum!}.
\item
  Exploit block patterns induced by quadray symmetries to stabilize
  metric inverses and reduce compute.
\item
  Combine integer-lattice projection with natural-gradient steps to
  balance discrete robustness and curvature-aware efficiency.
\item
  Purely discrete alternatives (e.g.,
  \passthrough{\lstinline!discrete\_ivm\_descent!}) provide monotone
  integer-valued descent when gradients are unreliable; hybrid schemes
  can interleave discrete steps with curvature-aware continuous
  proposals.
\end{itemize}

\hypertarget{implications-for-optimization-and-estimation}{%
\subsection{Implications for Optimization and
Estimation}\label{implications-for-optimization-and-estimation}}

\hypertarget{clarifications-on-frequencytime-dimensions}{%
\subsubsection{Clarifications on ``frequency/time''
dimensions}\label{clarifications-on-frequencytime-dimensions}}

\begin{itemize}
\tightlist
\item
  Fuller's discussions often treat frequency/energy as an additional
  organizing dimension distinct from Euclidean coordinates. In our
  manuscript, we keep the shape/angle relations (Fuller.4D) separate
  from time/energy bookkeeping; when temporal evolution is needed, we
  use explicit trajectories and metric analogies (Einstein.4D) without
  conflating with Euclidean 4D objects (Coxeter.4D). This separation
  avoids category errors while preserving the intended interpretability.
\end{itemize}

\hypertarget{on-distance-based-tetravolume-formulas-clarification}{%
\subsubsection{On distance-based tetravolume formulas
(clarification)}\label{on-distance-based-tetravolume-formulas-clarification}}

\begin{itemize}
\tightlist
\item
  When volumes are computed from edge lengths, PdF and Cayley--Menger
  operate in Euclidean length space and are converted to IVM
  tetravolumes via the S3 factor. In contrast, the Gerald de Jong
  formula computes IVM tetravolumes natively, agreeing numerically with
  PdF/CM after S3 without explicit XYZ intermediates. Tom Ace's 5×5
  determinant sits in the same native camp as de Jong's method. See
  references under the methods section for links to Urner's code
  notebooks and discussion.
\end{itemize}

\hypertarget{symbolic-analysis-bridging-vs-native-results-linkage}{%
\subsubsection{Symbolic analysis (bridging vs native) (Results
linkage)}\label{symbolic-analysis-bridging-vs-native-results-linkage}}

\begin{itemize}
\item
  Exact (SymPy) comparisons confirm that CM+S3 and Ace 5×5 produce
  identical IVM tetravolumes on canonical small integer-quadray
  examples. See the bridging vs native comparison figure above and the
  manifest \passthrough{\lstinline!sympy\_symbolics.txt!} alongside
  \passthrough{\lstinline!bridging\_vs\_native.csv!} in
  \passthrough{\lstinline!quadmath/output/!}.
\item
  Curvature-aware optimizers: Kronecker-factored approximations (K-FAC)
  leverage structure in \passthrough{\lstinline!F!} to accelerate
  training and improve stability; see
  \href{https://arxiv.org/abs/1503.05671}{K-FAC (arXiv:1503.05671)}.
  Similar ideas apply when quadray structure induces separable blocks.
\item
  Model selection: eigenvalue spread of \passthrough{\lstinline!F!}
  provides a lens on parameter identifiability; near-zero modes suggest
  redundancies or over-parameterization.
\item
  Robust computation: lattice normalization in quadray space yields
  discrete plateaus that complement FIM-based scaling for numerically
  stable trajectories.
\end{itemize}

\hypertarget{community-ecosystem-and-validation}{%
\subsection{Community Ecosystem and
Validation}\label{community-ecosystem-and-validation}}

The extensive computational ecosystem around Quadrays and synergetic
geometry provides validation, pedagogical context, and practical
implementations that complement and extend the methods developed in this
manuscript. Cross-language implementations serve as independent
verification of algorithmic correctness while educational materials
demonstrate practical applications across diverse computational
environments. See the Resources section for comprehensive details on the
4dsolutions organization, cross-language implementations, educational
frameworks, and community platforms.

\end{document}
