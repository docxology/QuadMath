% Options for packages loaded elsewhere
\PassOptionsToPackage{unicode}{hyperref}
\PassOptionsToPackage{hyphens}{url}
\PassOptionsToPackage{dvipsnames,svgnames*,x11names*}{xcolor}
%
\documentclass[
  10pt,
]{article}
\usepackage{lmodern}
\usepackage{setspace}
\usepackage{amssymb,amsmath}
\usepackage{ifxetex,ifluatex}
\ifnum 0\ifxetex 1\fi\ifluatex 1\fi=0 % if pdftex
  \usepackage[T1]{fontenc}
  \usepackage[utf8]{inputenc}
  \usepackage{textcomp} % provide euro and other symbols
\else % if luatex or xetex
  \usepackage{unicode-math}
  \defaultfontfeatures{Scale=MatchLowercase}
  \defaultfontfeatures[\rmfamily]{Ligatures=TeX,Scale=1}
  \setmainfont[]{DejaVu Serif}
  \setmonofont[]{DejaVu Sans Mono}
\fi
% Use upquote if available, for straight quotes in verbatim environments
\IfFileExists{upquote.sty}{\usepackage{upquote}}{}
\IfFileExists{microtype.sty}{% use microtype if available
  \usepackage[]{microtype}
  \UseMicrotypeSet[protrusion]{basicmath} % disable protrusion for tt fonts
}{}
\makeatletter
\@ifundefined{KOMAClassName}{% if non-KOMA class
  \IfFileExists{parskip.sty}{%
    \usepackage{parskip}
  }{% else
    \setlength{\parindent}{0pt}
    \setlength{\parskip}{6pt plus 2pt minus 1pt}}
}{% if KOMA class
  \KOMAoptions{parskip=half}}
\makeatother
\usepackage{xcolor}
\IfFileExists{xurl.sty}{\usepackage{xurl}}{} % add URL line breaks if available
\IfFileExists{bookmark.sty}{\usepackage{bookmark}}{\usepackage{hyperref}}
\hypersetup{
  colorlinks=true,
  linkcolor=red,
  filecolor=red,
  citecolor=red,
  urlcolor=red,
  pdfcreator={LaTeX via pandoc}}
\urlstyle{same} % disable monospaced font for URLs
\usepackage[margin=1cm,top=1cm,bottom=1cm,left=1cm,right=1cm,includeheadfoot]{geometry}
\usepackage{listings}
\newcommand{\passthrough}[1]{#1}
\lstset{defaultdialect=[5.3]Lua}
\lstset{defaultdialect=[x86masm]Assembler}
\usepackage{graphicx}
\makeatletter
\def\maxwidth{\ifdim\Gin@nat@width>\linewidth\linewidth\else\Gin@nat@width\fi}
\def\maxheight{\ifdim\Gin@nat@height>\textheight\textheight\else\Gin@nat@height\fi}
\makeatother
% Scale images if necessary, so that they will not overflow the page
% margins by default, and it is still possible to overwrite the defaults
% using explicit options in \includegraphics[width, height, ...]{}
\setkeys{Gin}{width=\maxwidth,height=\maxheight,keepaspectratio}
% Set default figure placement to htbp
\makeatletter
\def\fps@figure{htbp}
\makeatother
\setlength{\emergencystretch}{3em} % prevent overfull lines
\providecommand{\tightlist}{%
  \setlength{\itemsep}{0pt}\setlength{\parskip}{0pt}}
\setcounter{secnumdepth}{3}
% Enable graphics inclusion and ensure figure numbering works
\usepackage{graphicx}
\renewcommand{\figurename}{Figure}

% Configure fonts for Unicode support with fallbacks
\usepackage{newunicodechar}
\newunicodechar{⁴}{\textsuperscript{4}}
\newunicodechar{₄}{\textsubscript{4}}

% Enhanced code block styling for better contrast and readability
\usepackage{fancyvrb}
\usepackage{xcolor}
\usepackage{listings}

% Define custom colors for code blocks
\definecolor{codebg}{RGB}{245, 245, 245}      % Light gray background
\definecolor{codeborder}{RGB}{200, 200, 200}  % Medium gray border
\definecolor{codefg}{RGB}{50, 50, 50}         % Dark gray text

% Configure Verbatim environment for inline code
\DefineVerbatimEnvironment{Verbatim}{Verbatim}{%
    fontsize=\small,
    frame=single,
    framerule=0.5pt,
    framesep=3pt,
    rulecolor=\color{codeborder},
    bgcolor=\color{codebg},
    fgcolor=\color{codefg}
}

% Configure code block styling
\DefineVerbatimEnvironment{Highlighting}{Verbatim}{%
    fontsize=\footnotesize,
    frame=single,
    framerule=0.5pt,
    framesep=5pt,
    rulecolor=\color{codeborder},
    bgcolor=\color{codebg},
    fgcolor=\color{codefg}
}

% Style inline code with \texttt
\renewcommand{\texttt}[1]{%
    \colorbox{codebg}{\color{codefg}\ttfamily #1}%
}

% Configure listings package for code blocks
\lstset{
    backgroundcolor=\color{codebg},
    basicstyle=\footnotesize\ttfamily\color{codefg},
    breakatwhitespace=false,
    breaklines=true,
    captionpos=b,
    commentstyle=\color{codefg},
    deletekeywords={...},
    escapeinside={\%*}{*)},
    extendedchars=true,
    frame=single,
    framerule=0.5pt,
    framesep=5pt,
    keepspaces=true,
    keywordstyle=\color{codefg},
    language=Python,
    morekeywords={*,...},
    numbers=left,
    numbersep=5pt,
    numberstyle=\tiny\color{codefg},
    rulecolor=\color{codeborder},
    showspaces=false,
    showstringspaces=false,
    showtabs=false,
    stepnumber=1,
    stringstyle=\color{codefg},
    tabsize=2,
    title=\lstname
}

% Override any Pandoc default lstset configurations
\AtBeginDocument{
    \lstset{
        backgroundcolor=\color{codebg},
        basicstyle=\footnotesize\ttfamily\color{codefg},
        frame=single,
        framerule=0.5pt,
        framesep=5pt,
        rulecolor=\color{codeborder},
        numbers=left,
        numbersep=5pt,
        numberstyle=\tiny\color{codefg}
    }
}

% Configure hyperref colors consistently
\AtBeginDocument{
% Override pandoc's hidelinks setting with consistent options
\hypersetup{
    colorlinks=true,
    allcolors=red,
    linkcolor=red,
    urlcolor=red,
    citecolor=red,
    filecolor=red,
    menucolor=red,
    linktoc=all
}
}

% Simple page break support for document structure
% Note: Page breaks are handled in the markdown generation, not here

\title{4D Namespaces: Coxeter.4D, Einstein.4D, Fuller.4D}
\author{Daniel Ari Friedman\\ ORCID: 0000-0001-6232-9096\\ Email: daniel@activeinference.institute}
\date{August 15, 2025}

\begin{document}
\maketitle

{
\hypersetup{linkcolor=black}
\setcounter{tocdepth}{3}
\tableofcontents
}
\setstretch{1.0}
\hypertarget{d-namespaces-coxeter.4d-einstein.4d-fuller.4d}{%
\section{4D Namespaces: Coxeter.4D, Einstein.4D,
Fuller.4D}\label{d-namespaces-coxeter.4d-einstein.4d-fuller.4d}}

In this section, we clarify the three internal meanings of ``4D,''
following a dot-notation that avoids cross-domain confusion. Each
namespace represents a distinct mathematical framework with specific
applications in our quadray-based computational system.

\hypertarget{coxeter.4d-euclidean-eux2074}{%
\subsection{Coxeter.4D (Euclidean
E⁴)}\label{coxeter.4d-euclidean-eux2074}}

\begin{itemize}
\tightlist
\item
  \textbf{Definition}: Standard E⁴ with orthogonal axes and Euclidean
  metric; the proper setting for classical regular polytopes. As Coxeter
  notes (Regular Polytopes, Dover ed., p.~119), this Euclidean 4D is not
  spacetime. Lattice/packing discussions connect to Conway \& Sloane's
  systematic treatment of higher-dimensional sphere packings and
  lattices
  (\href{https://link.springer.com/book/10.1007/978-1-4757-6568-7}{Sphere
  Packings, Lattices and Groups (Springer)}).
\item
  \textbf{Usage}: Embed Quadray configurations or compare alternative
  parameterizations when a strictly Euclidean 4D setting is desired.
\item
  \textbf{Simplexes}: Simplex structures extend naturally to 4D and
  beyond (e.g., pentachora).
\item
  \textbf{Mathematical context}: This framework is appropriate for
  standard Euclidean geometry, including the Cayley-Menger determinant
  for computing volumes from edge lengths.
\end{itemize}

\hypertarget{einstein.4d-relativistic-spacetime}{%
\subsection{Einstein.4D (Relativistic
spacetime)}\label{einstein.4d-relativistic-spacetime}}

\textbf{Definition}: Minkowski spacetime with indefinite metric
signature, representing the geometric framework for special relativity.
This namespace provides the mathematical foundation for understanding
space-time relationships and relativistic phenomena.

\begin{itemize}
\item
  \textbf{Spacetime}: Minkowski metric signature.
\item
  \textbf{Line element} (mostly-plus convention; see
  \href{https://en.wikipedia.org/wiki/Minkowski_space}{Minkowski
  space}):

  \begin{equation}\label{eq:einstein_line_element}
  ds^2 = -c^2\,dt^2 + dx^2 + dy^2 + dz^2
  \end{equation}
\item
  \textbf{Optimization analogy}: Metric-aware geodesics generalize to
  information geometry where the Fisher metric replaces the physical
  metric. See
  \href{https://en.wikipedia.org/wiki/Fisher_information}{Fisher
  information} and
  \href{https://en.wikipedia.org/wiki/Natural_gradient}{natural
  gradient}.
\item
  \textbf{Important note}: This namespace is used ONLY as a
  metric/geodesic analogy when discussing information geometry. Physical
  constants G, c, Λ do not appear in Quadray lattice methods and should
  not be mixed with IVM unit conventions.
\end{itemize}

\hypertarget{fuller.4d-synergetics-quadrays}{%
\subsection{Fuller.4D (Synergetics /
Quadrays)}\label{fuller.4d-synergetics-quadrays}}

\textbf{Definition}: Tetrahedral coordinate system based on four
non-negative components representing directions to the vertices of a
regular tetrahedron from its center. This namespace embodies the
synergetic approach to geometry, emphasizing shape relationships and
integer tetravolumes within the IVM framework.

\begin{itemize}
\tightlist
\item
  \textbf{Basis}: Four non-negative components A,B,C,D with at least one
  zero post-normalization, treated as a vector (direction and
  magnitude), not merely a point. Overview:
  \href{https://en.wikipedia.org/wiki/Quadray_coordinates}{Quadray
  coordinates}.
\item
  \textbf{Geometry}: Tetrahedral; unit tetrahedron volume = 1; integer
  lattice aligns with close-packed spheres (IVM). Background:
  \href{https://en.wikipedia.org/wiki/Synergetics_(Fuller)}{Synergetics}.
\item
  \textbf{Distances}: Computed via appropriate projective normalization;
  edges align with tetrahedral axes. The IVM = CCP = FCC shortcut allows
  working in 3D embeddings for visualization while preserving the
  underlying Fuller.4D tetrahedral accounting.
\item
  \textbf{Implementation heritage}: Extensive computational validation
  through Kirby Urner's
  \href{https://github.com/4dsolutions}{4dsolutions ecosystem}. See the
  \href{07_resources.md}{Resources} section for comprehensive details on
  computational implementations and educational materials.
\end{itemize}

\hypertarget{directions-not-dimensions-language-and-models}{%
\subsubsection{Directions, not dimensions (language and
models)}\label{directions-not-dimensions-language-and-models}}

\begin{itemize}
\tightlist
\item
  \textbf{Vector-first framing}: Treat Quadrays as four canonical
  directions (``spokes'' to the vertices of a regular tetrahedron from
  its center), not as four orthogonal dimensions. The methane molecule
  (CH₄) and caltrop shape are helpful mental models.
\item
  \textbf{Origins outside Synergetics}: Quadrays did not originate with
  Fuller; we adopt the coordinate system within the IVM context. See
  \href{https://en.wikipedia.org/wiki/Quadray_coordinates}{Quadray
  coordinates}.
\item
  \textbf{Language games}: Quadrays and Cartesian are parallel vector
  languages on the same Euclidean container; teaching them together
  avoids oscillating between ``points now, vectors later.''
\end{itemize}

\hypertarget{figures}{%
\subsubsection{Figures}\label{figures}}

\begin{figure}
\centering
\includegraphics{../output/figures/ivm_neighbors_edges.png}
\caption{\textbf{IVM neighbors and coordination patterns (2×2 panel
layout)}. \textbf{Panel A}: The twelve nearest IVM neighbors plotted as
blue points in 3D space under the default embedding, showing the
positions corresponding to permutations of the Quadray integer
coordinates \{2,1,1,0\}. These points form the vertices of a
cuboctahedron (vector equilibrium) centered at the origin with uniform
radial distances. \textbf{Panel B}: The same neighbor points with radial
edges (light lines) connecting each neighbor to the central origin,
emphasizing the spoke-like radial symmetry and equal distances from
center to shell. \textbf{Panel C}: Twelve-around-one close-packed
spheres configuration where each neighbor position hosts a sphere with
radius chosen so neighboring spheres kiss along cuboctahedron edges,
illustrating the fundamental CCP/FCC/IVM correspondence. The central
gray sphere represents the ``one'' in Fuller's ``twelve around one''
motif. \textbf{Panel D}: Adjacency graph showing strut connections
(solid lines) between touching neighbor spheres, revealing the
cuboctahedron's edge structure, plus light radial cables to the origin
representing a stylized tensegrity interpretation of the vector
equilibrium geometry.}
\end{figure}

\begin{figure}
\centering
\includegraphics{../output/figures/quadray_clouds.png}
\caption{\textbf{Random Quadray point clouds under different embeddings
(3-panel comparison)}. Each panel shows 200 randomly sampled integer
Quadray coordinates with components in \{0,1,2,3,4,5\} projected to 3D
space using different embedding matrices. \textbf{Left panel (Default
embedding)}: Points (blue) under the default symmetric embedding matrix
showing the natural tetrahedral-symmetric distribution of normalized
Quadrays in 3D space. \textbf{Center panel (Scaled embedding, 0.75×)}:
The same Quadray points (orange) under a uniformly scaled version of the
default embedding, demonstrating how the point cloud structure scales
proportionally while preserving relative geometries. \textbf{Right panel
(Urner embedding)}: The same points (purple) projected through the
canonical Urner embedding matrix, illustrating how different linear
mappings from Fuller.4D to Coxeter.4D (3D slice) affect the spatial
distribution while preserving the underlying discrete lattice
relationships. This comparison demonstrates the flexibility in choosing
embeddings for visualization and analysis while maintaining the
fundamental Quadray coordinate relationships.}
\end{figure}

In the previous figure, we show the twelve nearest IVM neighbors with
coordination patterns and vector equilibrium geometry; the current
figure illustrates random Quadray clouds under several embeddings.

Vector equilibrium (cuboctahedron). The shell formed by the 12 nearest
IVM neighbors is the cuboctahedron, also called the vector equilibrium
in synergetics. All 12 vertices are equidistant from the origin with
equal edge lengths, modeling a balanced local packing. This geometry
underlies the ``twelve around one'' close-packing motif and appears in
tensegrity discussions as a canonical balanced structure. See
background:
\href{https://en.wikipedia.org/wiki/Cuboctahedron}{Cuboctahedron (vector
equilibrium)} and synergetics references. Computational demonstrations
include related visualizations in the 4dsolutions ecosystem. See the
\href{07_resources.md}{Resources} section for comprehensive details.

\hypertarget{clarifying-remarks}{%
\subsubsection{Clarifying remarks}\label{clarifying-remarks}}

\begin{itemize}
\tightlist
\item
  ``A time machine is not a tesseract.''
  \href{https://groups.io/g/synergeo/topic/my_take_on_close_pack/114531919}{KU
  on synergeo} The tesseract is a Euclidean 4D object (Coxeter.4D),
  while Minkowski spacetime (Einstein.4D) is indefinite and not
  Euclidean; conflating the two leads to category errors. Fuller.4D, in
  turn, is a tetrahedral, mereological framing of ordinary space
  emphasizing shape/angle relations and IVM quantization. Each namespace
  carries distinct assumptions and should be used accordingly in
  analysis.
\end{itemize}

\hypertarget{practical-usage-guide}{%
\subsection{Practical usage guide}\label{practical-usage-guide}}

\begin{itemize}
\tightlist
\item
  Use \textbf{Fuller.4D} when working with Quadrays, integer
  tetravolumes, and IVM neighbors (native lattice calculations).
\item
  Use \textbf{Coxeter.4D} for Euclidean length-based formulas,
  higher-dimensional polytopes, or comparisons in E⁴ (including
  Cayley--Menger).
\item
  Use \textbf{Einstein.4D} as a metric analogy when discussing geodesics
  or time-evolution; do not mix with synergetic unit conventions.
\end{itemize}

\end{document}
