% Options for packages loaded elsewhere
\PassOptionsToPackage{unicode}{hyperref}
\PassOptionsToPackage{hyphens}{url}
\PassOptionsToPackage{dvipsnames,svgnames*,x11names*}{xcolor}
%
\documentclass[
  10pt,
]{article}
\usepackage{lmodern}
\usepackage{setspace}
\usepackage{amssymb,amsmath}
\usepackage{ifxetex,ifluatex}
\ifnum 0\ifxetex 1\fi\ifluatex 1\fi=0 % if pdftex
  \usepackage[T1]{fontenc}
  \usepackage[utf8]{inputenc}
  \usepackage{textcomp} % provide euro and other symbols
\else % if luatex or xetex
  \usepackage{unicode-math}
  \defaultfontfeatures{Scale=MatchLowercase}
  \defaultfontfeatures[\rmfamily]{Ligatures=TeX,Scale=1}
  \setmainfont[]{DejaVu Serif}
  \setmonofont[]{DejaVu Sans Mono}
\fi
% Use upquote if available, for straight quotes in verbatim environments
\IfFileExists{upquote.sty}{\usepackage{upquote}}{}
\IfFileExists{microtype.sty}{% use microtype if available
  \usepackage[]{microtype}
  \UseMicrotypeSet[protrusion]{basicmath} % disable protrusion for tt fonts
}{}
\makeatletter
\@ifundefined{KOMAClassName}{% if non-KOMA class
  \IfFileExists{parskip.sty}{%
    \usepackage{parskip}
  }{% else
    \setlength{\parindent}{0pt}
    \setlength{\parskip}{6pt plus 2pt minus 1pt}}
}{% if KOMA class
  \KOMAoptions{parskip=half}}
\makeatother
\usepackage{xcolor}
\IfFileExists{xurl.sty}{\usepackage{xurl}}{} % add URL line breaks if available
\IfFileExists{bookmark.sty}{\usepackage{bookmark}}{\usepackage{hyperref}}
\hypersetup{
  colorlinks=true,
  linkcolor=red,
  filecolor=red,
  citecolor=red,
  urlcolor=red,
  pdfcreator={LaTeX via pandoc}}
\urlstyle{same} % disable monospaced font for URLs
\usepackage[margin=0.7cm,top=0.7cm,bottom=0.7cm,left=0.7cm,right=0.7cm,includeheadfoot]{geometry}
\setlength{\emergencystretch}{3em} % prevent overfull lines
\providecommand{\tightlist}{%
  \setlength{\itemsep}{0pt}\setlength{\parskip}{0pt}}
\setcounter{secnumdepth}{3}
% Enable graphics inclusion and ensure figure numbering works
\usepackage{graphicx}
\renewcommand{\figurename}{Figure}

% Configure fonts for Unicode support with fallbacks
\usepackage{newunicodechar}
\newunicodechar{⁴}{\textsuperscript{4}}
\newunicodechar{₄}{\textsubscript{4}}

% Configure hyperref colors consistently
\AtBeginDocument{
% Override pandoc's hidelinks setting with consistent options
\hypersetup{
    colorlinks=true,
    allcolors=red,
    linkcolor=red,
    urlcolor=red,
    citecolor=red,
    filecolor=red,
    menucolor=red,
    linktoc=all
}
}

\title{QuadMath: Front Matter and Abstract}
\author{Daniel Ari Friedman\\ ORCID: 0000-0001-6232-9096\\ Email: daniel@activeinference.institute}
\date{August 14, 2025}

\begin{document}
\maketitle

{
\hypersetup{linkcolor=red}
\setcounter{tocdepth}{3}
\tableofcontents
}
\setstretch{1.0}
\hypertarget{quadmath-an-analytical-review-of-4d-and-quadray-coordinates}{%
\section{QuadMath: An Analytical Review of 4D and Quadray
Coordinates}\label{quadmath-an-analytical-review-of-4d-and-quadray-coordinates}}

\hypertarget{abstract}{%
\subsection{Abstract}\label{abstract}}

We review a unified analytical framework for four dimensional (4D)
modeling and Quadray coordinates, synthesizing geometric foundations,
optimization on tetrahedral lattices, and information geometry. Building
on R. Buckminster Fuller's synergetics
\href{https://en.wikipedia.org/wiki/Synergetics_(Fuller)}{Synergetics
(Fuller)} and the Quadray coordinate system, with extensive reference to
Kirby Urner's computational implementations across multiple programming
languages (see the comprehensive
\href{https://github.com/4dsolutions}{4dsolutions ecosystem} including
Python, Rust, Clojure, and POV-Ray implementations), we review how
integer lattice constraints yield integer volume quantization of
tetrahedral simplexes, creating discrete ``energy levels'' that
regularize optimization and enable integer-based optimization. We adapt
standard methods (e.g.,
\href{https://en.wikipedia.org/wiki/Nelder\%E2\%80\%93Mead_method}{Nelder--Mead
method}) to the quadray lattice, define
\href{https://en.wikipedia.org/wiki/Fisher_information}{Fisher
information} in Quadray parameter space, and analyze optimization as
geodesic motion on an information manifold via the
\href{https://en.wikipedia.org/wiki/Natural_gradient}{natural gradient}.
We also integrate three distinct 4D namespaces---Coxeter.4D (Euclidean
E⁴), Einstein.4D (Minkowski spacetime), and Fuller.4D
(synergetics/Quadrays)---develop analytical tools and equations, and
survey extensions and applications across AI,
\href{https://welcome.activeinference.institute/}{active inference},
cognitive security, and complex systems. The result is a cohesive,
interpretable approach for robust, geometry-grounded computation in 4D.

Keywords: Quadray coordinates, 4D geometry, tetrahedral lattice, integer
volume quantization, information geometry, optimization, synergetics,
active inference.

\hypertarget{manuscript-structure}{%
\subsection{Manuscript structure}\label{manuscript-structure}}

\begin{itemize}
\tightlist
\item
  Introduction: motivates Quadrays, clarifies 4D namespaces (Coxeter.4D,
  Einstein.4D, Fuller.4D), and summarizes contributions.
\item
  Methods: details coordinate conventions, exact tetravolumes,
  conversions, and lattice-aware optimization methods (Nelder--Mead and
  discrete IVM descent).
\item
  Results: empirical comparisons and demonstrations are shown inline and
  saved under \texttt{quadmath/output/} (PNG/CSV/NPZ/MP4) for
  reproducibility.
\item
  Discussion: interprets results, limitations, and implications;
  outlines future work.
\item
  Appendices: equations, free-energy background, and a consolidated
  symbols/glossary with an auto-generated API index.
\end{itemize}

\hypertarget{reproducibility-and-data-availability}{%
\subsection{Reproducibility and data
availability}\label{reproducibility-and-data-availability}}

\begin{itemize}
\tightlist
\item
  The manuscript Markdown and code to generate the PDF are available on
  the project repository (\texttt{QuadMath} on GitHub, @docxology
  username). See the repository home page for source, figures, and
  scripts: \href{https://github.com/docxology/quadmath}{QuadMath
  repository}.
\item
  The manuscript is licensed under the Apache License 2.0. See the
  \href{../LICENSE}{LICENSE} file for details.
\item
  The manuscript is accompanied by a fully-tested Python codebase under
  \texttt{src/} with unit tests under \texttt{tests/}, complemented by
  extensive cross-validation against Kirby Urner's reference
  implementations in the
  \href{https://github.com/4dsolutions}{4dsolutions ecosystem}.
\item
  All figures referenced in the manuscript are generated by scripts
  under \texttt{quadmath/scripts/} and saved to
  \texttt{quadmath/output/} with lightweight CSV/NPZ alongside images.
\item
  Tests accompany all methods under \texttt{src/} and enforce 100\%
  coverage for \texttt{src/}; external validation includes comparisons
  with
  \href{https://github.com/4dsolutions/m4w/blob/main/qrays.py}{\texttt{qrays.py}}
  and
  \href{https://github.com/4dsolutions/m4w/blob/main/tetravolume.py}{\texttt{tetravolume.py}}
  algorithms.
\item
  Symbols and notation are standardized across sections; see Appendix:
  Symbols and Glossary for a consolidated table of variables and
  constants used throughout. Equation labels (e.g., Eq.
  \eqref{eq:lattice_det} and Eq. \eqref{eq:supp_fim}) and figure labels
  (e.g., Figure \ref{fig:volumes_scale}) are used consistently.
\end{itemize}

\end{document}
